%CLASSE DOCUMENTO - LINGUA E DIMENSIONE FONT
\documentclass[a4paper, 
				11pt,
				stile=classica,
				tipotesi=triennale]{toptesi}

%%%%%%%%%%%%%%%%%%%%%%%%%%%%%%%%%%%%%%%%%%%%%%%%%%%%%%%%%%%%%%%

% INCLUSIONE PACCHETTI
\usepackage[utf8]{inputenc} %utf8
\usepackage[italian]{babel}
\usepackage[T1]{fontenc}
\usepackage{blindtext}
\usepackage{graphicx,wrapfig}
\usepackage{booktabs}
\usepackage{lmodern}
\usepackage{varioref}
\usepackage{url}
\usepackage{array}
\usepackage{paralist}{\obeyspaces\global\let =\space}
\usepackage{verbatim} 
\usepackage{subfig}
\usepackage{tabularx}
\usepackage{amsmath}
\usepackage{amsfonts}
\usepackage{float}
\usepackage{amssymb}
\usepackage{multicol}
\usepackage{multirow}
\usepackage{listings}
\usepackage[pass]{geometry}
\usepackage[figuresright]{rotating}
\usepackage{algorithm}
\usepackage{algorithmic}
\usepackage{amsmath}
\usepackage[babel]{csquotes}
\usepackage{hyperref}
\usepackage[backend=bibtex]{biblatex}
%%%%%%%%%%%%%%%%%%%%%%%%%%%%%%%%%%%%%%%%%%%%%%%%%%%%%%%%%%%%%%%


% CONFIGURAZIONE LINK E RIFERIMENTI
\hypersetup{%
    pdfpagemode={UseOutlines},
    bookmarksopen,
    pdfstartview={FitH},
    colorlinks,
    linkcolor={black}, %COLORE DEI RIFERIMENTI AL TESTO
    citecolor={blue}, %COLORE DEI RIFERIMENTI ALLE CITAZIONI
    urlcolor={blue} %COLORI DEGLI URL
}

%%%%%%%%%%%%%%%%%%%%%%%%%%%%%%%%%%%%%%%%%%%%%%%%%%%%%%%%%%%%%%%

% CONFIGURAZIONE LISTATI/CODICE - CANCELLARE SE NON NECESSARIO
% PYTHON - BIANCO E NERO
\lstset{%
	captionpos=b,
	language=Java,
	basicstyle =\small\ttfamily,
	keywordstyle=\color{black}\bfseries,
	breaklines=true,
	breakatwhitespace=true,
	frame=lines,
	numbers=left,
	numberstyle=\footnotesize
}

\lstdefinelanguage{Cypher}{
    keywords={Match, retrun, MATCH, RETURN, Match, Return},
    sensitive=true,
    morecomment=[l]{//},
    morecomment=[s]{/*}{*/}
}

\lstset{
	captionpos=b,
	language=Cypher,
	basicstyle =\small\ttfamily,
	keywordstyle=\color{black}\bfseries,
	breaklines=true,
	breakatwhitespace=true,
	frame=lines,
	numbers=left,
	numberstyle=\footnotesize
}

\lstdefinelanguage{HTML}{
    keywords={<, >, div, p, label, table, class, />, <div, </div>, {{, }}, h1, h2},
    sensitive=true,
    morecomment=[l]{//},
    morecomment=[s]{<!--}{-->}
}

\lstset{
	captionpos=b,
	language=HTML,
	basicstyle =\small\ttfamily,
	keywordstyle=\color{black}\bfseries,
	breaklines=true,
	breakatwhitespace=true,
	frame=lines,
	numbers=left,
	numberstyle=\footnotesize
}

\lstdefinelanguage{Javascript}{
    keywords={var, const, =, {, }, :},
    sensitive=true,
    morecomment=[l]{//},
    morecomment=[s]{/*}{*/}
}

\lstset{
	captionpos=b,
	language=Javascript,
	basicstyle =\small\ttfamily,
	keywordstyle=\color{black}\bfseries,
	breaklines=true,
	breakatwhitespace=true,
	frame=lines,
	numbers=left,
	numberstyle=\footnotesize
}

%%%%%%%%%%%%%%%%%%%%%%%%%%%%%%%%%%%%%%%%%%%%%%%%%%%%%%%%%%%%%%%

% FRENCHSPACING VA _SEMPRE_ ABILITATO PER DOCUMENTI IN ITALIANO
\frenchspacing

%%%%%%%%%%%%%%%%%%%%%%%%%%%%%%%%%%%%%%%%%%%%%%%%%%%%%%%%%%%%%%%

%Rinominato il menu dei listati
\renewcommand\lstlistingname{Listato}
\renewcommand\lstlistlistingname{Elenco dei listati}

%Aggiunta numerazione sottocapitoli
\setcounter{tocdepth}{3}
\setcounter{secnumdepth}{4} 
%%%%%%%%%%%%%%%%%%%%%%%%%%%%%%%%%%%%%%%%%%%%%%%%%%%%%%%%%%%%%%%

%DEFINIZIONE SEZIONI IN NUMERAZIONE ROMANA
%ELENCO DEI LISTATI/CODICI
\makeatletter
\newcommand\listofcodes{%
 \iffrontmatter\else\frontmattertrue\fi
 \if@openright\cleardoublepage\else\clearpage\fi
 % change the meaning of \chapter in a group
 \begingroup\def\chapter##1{\@schapter}
 \phantomsection % for the hyperlink
 \lstlistoflistings 
 \endgroup
} 
\makeatother

%%%%%%%%%%%%%%%%%%%%%%%%%%%%%%%%%%%%%%%%%%%%%%%%%%%%%%%%%%%%%%%

% INFORMAZIONI PDF - PERSONALIZZARE
\pdfinfo{%
  /Title    (PROGETTAZIONE E SVILUPPO DI UN SISTEMA DISTRIBUITO PER L'ANALISI DELLE TRANSAZIONI BITCOIN)
  /Author   (Antonio Riccardi)
  /Keywords (bitcoin;blockchain;spark;neo4j;nodejs)
}


% LISTA DEI CAPITOLI DA INCLUDERE - PERSONALIZZARE
\includeonly{%
intro,%
overview,%
implementazione,%
conclusioni,%
%app_a,%
}

% FILE DI BIBLIOGRAFIA
%\bibliography{bibliography} 
\bibliography{bibliography,webbib}

%\defbibheading{cartaceo}{\section*{Cartacei}}
%\defbibheading{web}{\section*{Sitografia}}

\defbibheading{cartaceo}{\subsection*{Manuali cartacei}}
\defbibheading{web}{\subsection*{Siti Web consultati}}

%%%%%%%%%%%%%%%%%%%%%%%%%%%%%%%%%%%%%%%%%%%%%%%%%%%%%%%%%%%%%%%

% INIZIO DOCUMENTO
\begin{document}


\begin{ThesisTitlePage}
    % FRONTESPIZIO - PERSONALIZZARE
	% ELIMINATE LE VOCI CHE NON VI SERVONO

	% UNIVERSITA - NOME
	\ateneo{Universit\`a degli Studi di Napoli "Parthenope"}

	% FACOLTA - DICITURA - CANCELLARE O DECOMMENTARE
	%\FacoltaDi{Faculty of}
	% FACOLTA - NOME
	\nomeateneo{Dipartimento di Scienze e Tecnologie}

	% CORSO DI LAUREA - DICITURA (MANTENERE LO SPAZIO) - CANCELLARE O DECOMMENTARE
	%\CorsoDiLaureaIn{Master of Science in }
	% CORSO DI LAUREA - NOME
	\corsodistudi{Informatica}


	% TIPOLOGIA TESI
	\NomeElaborato{Tesi di Laurea Triennale}

	% TITOLO
	\titolo{PROGETTAZIONE E SVILUPPO DI UN SISTEMA DISTRIBUITO PER L'ANALISI DELLE TRANSAZIONI BITCOIN}

	% RELATORE/I - DICITURA - CANCELLARE SE UN SOLO RELATORE
	%\AdvisorName{Relatori}
	% RELATORE - PROF. NOME E COGNOME
	\relatore{prof. Alessio Ferone}


	% CANDIDATO - NOME E COGNOME
	\candidato{Antonio Riccardi \\mat. 0124000411}

	% LOGO UNIVERSITA
	\logosede{images/logo}

	% DATA - MESE ANNO
	\sedutadilaurea{Anno accademico 2017 - 2018}
    
\end{ThesisTitlePage}

%INTERLINEA - DEFAULT 1 - NON ESAGERATE, NON SUPERATE MAI 1.3 ;)
\interlinea{1.0}

%%%%%%%%%%%%%%%%%%%%%%%%%%%%%%%%%%%%%%%%%%%%%%%%%%%%%%%%%%%%%%%

\frontmatter

% DEDICA - PERSONALIZZARE
% VSPACE - PROPORZIONE USATA PER CENTRATURA VERTICALE DEL TESTO
% FLUSHRIGHT - ALLINEAMENTO ORIZZONTALE A DESTRA
\vspace*{\stretch{1}}
\begin{flushright}
\noindent
	Ai miei genitori.
\end{flushright}
\vspace*{\stretch{6}}
\cleardoublepage


%%%%%%%%%%%%%%%%%%%%%%%%%%%%%%%%%%%%%%%%%%%%%%%%%%%%%%%%%%%%%%%

% RINGRAZIAMENTI - PERSONALIZZARE
\ringraziamenti
Le  persone  che  mi  conoscono  sanno  che  esprimere  i  miei  sentimenti  è  sempre  stato  difficile,  ma  in  questo  caso  non  posso  farne  a  meno  di  ringraziare  chi  mi  ha  sempre  sostenuto  ed  incoraggiato  per  questi  lunghi  otto  anni.    I  miei  più  sentiti  ringraziamenti,  quindi,  vanno  a  tutti  i  miei  amici  che  nel  bene  e  nel  male  sono  sempre  stati  lì,  pronti  a  supportarmi  e  sopportarmi  in  questa  lunga  carriera  universitaria.
\\In  particolare,  dico  GRAZIE  a  Remo  con  la  quale  ho  condiviso  questa  lunga  avventura  sin  dall'inizio;  a  Poldo  che  nei  periodi  più  bui  è  stato  lì  ad  incitarmi  ed  incoraggiarmi  ad  andare  oltre  l'ostacolo;  a  Vincenzo  che  con  la  sua  gentilezza  e  disponibilità  ha  cercato  in  ogni  modo  di  aiutarmi  a  superare  gli  ultimi  esami;  a  Giusina  che  con  i  suoi  modi "acerrani" ha  sempre  spinto  a  fare  di  meglio;  all'amico  di  una  vita,  Giovanni  Cirella,  che  è  sempre  stato,  sin  da  piccoli,  dalla  mia  parte; ai miei colleghi-amici che in questi quattro anni hanno subito le mie lamentele; ai  miei  fratelli  Rosa,  Giovanni  e  Carmela  che  hanno  sempre  festeggiato  per  ogni  mio  esame  e  rincuorato  quando  invece  non  andava  bene,  facendomi  vivere  la  carriera  universitaria  senza  pressioni.
\\Infine,  voglio  dire  GRAZIE  ed  un  TI  AMO  alla  persona  più  importante  che  ho  conosciuto  in  questa  lunga  carriera  universitaria,  quella  che  mi  è  sempre  stata  vicina  e  che  mi  accompagnerà  anche  oltre  questo  percorso  diventando  mia  moglie:  ALESSIA.

%%%%%%%%%%%%%%%%%%%%%%%%%%%%%%%%%%%%%%%%%%%%%%%%%%%%%%%%%%%%%%%

% ABSTRACT - PERSONALIZZARE
\sommario
La seguente tesi ha come obiettivo la realizzazione di un'applicazione distribuita in grado di fare analisi di transazioni provenienti dalla blockchain di Bitcoin.\\Il nocciolo della questione affrontata è quello di studiare la struttura di ogni singola piattaforma distribuita e di implementare componenti in grado di ricavare informazioni utili per l'analisi delle transazioni. In particolare, ci si soffermerà sulle principali tecnologie informatiche che permettono di costruire sistemi distribuiti, come Spark ed Hadoop, e sulle strutture dati che facilitano l'immagazzinamento delle informazioni come Neo4j.
\\Nella prima parte della tesi si procede con l'introduzione alla tecnologia alla base della moneta virtuale Bitcoin, facendo luce su cos'è la Blockchain e quali vantaggi/svantaggi presenta rispetto ad altri sistemi che gestiscono transazioni monetarie. A seguire, sono presi in esame le principali soluzioni di analisi di transazioni già esistenti. 
\\Il secondo capitolo dell'elaborato mostra una panoramica del disegno dell'architettura del sistema realizzato, mettendo in luce i principali componenti del sistema. A tal fine è presente un'ampia e dettagliata descrizione di ogni singolo componente del sistema specificando i vantaggi ottenuti dal proprio utilizzo.
\\Il terzo capitolo descrive come è stata fatta l'implementazione del sistema, mostrando porzioni di codice e screen dell'applicazione che ci aiutano nella comprensione di come funziona il sistema.
\\Infine nell'ultimo capitolo vengono trattate le considerazioni finali del lavoro svolto soffermandosi anche sui possibili sviluppi. 

%%%%%%%%%%%%%%%%%%%%%%%%%%%%%%%%%%%%%%%%%%%%%%%%%%%%%%%%%%%%%%%


% INDICI - ELIMINARE GLI INDICI NON NECESSARI

% INDICE GENERALE
\tableofcontents

% INDICE DELLE FIGURE
\listoffigures

% INDICE DELLE TABELLE
\listoftables

% INDICE DEI CODICI

\listofcodes
\addcontentsline{toc}{chapter}{Elenco dei listati}
%%%%%%%%%%%%%%%%%%%%%%%%%%%%%%%%%%%%%%%%%%%%%%%%%%%%%%%%%%%%%%%

\mainmatter
\setcounter{secnumdepth}{4}

% INCLUSIONE FILE CAPITOLI - PERSONALIZZARE - TENERE COERENTE CON LISTA IN ALTO
\chapter{Bitcoin: Fonte di bigdata}
\label{chap:bitcoin fonte di bigdata}

Nel lontano 2009 uno pseudonimo Satoshi Nakamoto pubblicò il manifesto \textit{"Bitcoin: A Peer-to-Peer Electronic Cash System"}\cite{bitcoin:white-paper} sancendo la nascita di una nuova moneta digitale chiamata Bitcoin, che fa  del suo punto di forza la libertà e la sicurezza. Nel corso degli anni il valore della moneta è cresciuto sino a raggiungere picchi di 19290 dollari nel 17 Dicembre 2017 \cite{blockchain.com:valueOf}. Di pari passo, anche il numero di transazioni giornaliere sulla rete bitcoin ha raggiunto cifre esorbitanti riuscendo a validare 5 transazioni al secondo\cite{blockchain.com:transactions}, questa enorme mole di dati è possibile processarla solo con sistemi distribuiti che gestiscono i Big Data. Quest'ultimi rappresentano tutti quei dati che possono essere disponibili in enormi volumi, possono presentarsi con formati semistrutturati o addirittura destrutturati e possono essere prodotti con estrema velocità. Uno degli aspetti che caratterizzano i big data è la loro quantità. Questi dati infatti, sono generati dall'utente attraverso gli strumenti del Web 2.0, sistemi gestionali, oppure generati automaticamente da macchine (sensori, sistemi di calcolo) assumendo volumi rilevanti, non più gestibili con strumenti di database tradizionali. Non a caso l'elaborato di tesi, utilizza un sistema distribuito creato ad hoc per i big data per gestire l'enorme volume di transazioni che circolano nella rete Bitcoin.
\\Bitcoin, oltre all'impatto economico, ha avuto un importante ruolo nel campo della ricerca, donandoci un sistema che gestisce le transazioni tra due parti in un network peer-to-peer, diversamente dai sistemi tradizionali, denominato Blockchain.

\section{Blockchain}
\label{sec:blockchain}

La blockchain come dice il nome è una catena di blocchi che implementano un database  aperto e distribuito, atto a memorizzare le transazioni tra due parti in modo sicuro, verificabile e permanente. In altre parole, la blockchain rappresenta il libro contabile (o libro mastro), ossia il registro sul quale sono riportati tutti gli scambi tra le parti. Questo libro mastro è distribuito (Distributed Ledger) replicato e sincronizzato tra tutti i membri di una rete. In questo database vengono registrate le transazioni (come lo scambio di beni o informazioni) tra i partecipanti alla rete.
\\I dati non sono memorizzati su un solo computer ma su più macchine collegate tra loro attraverso una rete peer-to-peer [\ref{fig:blockchainNetwork}] sottoforma di blocchi.
\\Ciascun nodo è autorizzato ad aggiornare e gestire il libro contabile distribuito in modo indipendente, ma sotto il controllo consensuale degli altri nodi. Infatti, gli aggiornamenti non sono più gestiti, come accadeva tradizionalmente, sotto il controllo rigoroso di un’autorità centrale, ma sono invece creati e caricati da ciascun nodo in modo appunto indipendente. In questo modo ogni partecipante è in grado di processare e controllare ogni transazione ma, nello stesso tempo ogni singola transazione, essendo gestita in autonomia, deve essere verificata, crittografata e approvata dalla maggioranza dei partecipanti alla rete. 

\begin{figure}[H]
	\centering
	\includegraphics[width=0.5\textwidth]{images/blockchainNetwork.png}
	\caption{Rete distribuita di nodi paritari.}
	\label{fig:blockchainNetwork}
\end{figure}

Ogni blocco contenuto nella blochchain archivia un insieme di transazioni validate correlate da un Marcatore Temporale (Timestamp) ed un hash (una stringa alfanumerica) che identifica il blocco in modo univoco e che permette il collegamento con il blocco precedente[\ref{fig:blockChain}].

\begin{figure}[H]
	\centering
	\includegraphics[width=\textwidth]{images/blockChain.png}
	\caption{Catena di blocchi nella blockchain.}
	\label{fig:blockChain}
\end{figure}

Perché un nuovo blocco di transazioni sia aggiunto alla Blockchain è necessario appunto che sia controllato, validato e crittografato. Solo con questo passaggio può poi diventare attivo ed essere aggiunto alla Blockchain. Per effettuare questo passaggio è necessario che ogni volta che viene composto un blocco venga risolto un complesso problema matematico che richiede un cospicuo impegno anche in termini di potenza e di capacità elaborativa. Questa operazione viene definita come “Mining” ed è svolta dai “Miner”. La risoluzione del problema è un processo irreversibile per cui il blocco valido aggiunto alla catena diviene immutabile.
\\In conclusione, questa tecnologia può essere utilizzata in tutti gli ambiti in cui è necessaria una relazione tra più persone o gruppi. Ad esempio può garantire il corretto scambio di titoli e azioni o sostituire un atto notarile, perché ogni transazione viene sorvegliata da una rete di nodi che ne garantiscono la correttezza senza l’ausilio di intermediari.
\\Bitcoin sfrutta questa tecnologia per gestire le transazioni finanziarie in modo sicuro e del tutto autonomo.
\section{Bitcoin}
\label{sec:bitcoin}
Bitcoin (codice: BTC o XBT) è una criptovaluta e un sistema di pagamento mondiale creato nel 2009 da un anonimo inventore, noto con lo pseudonimo di Satoshi Nakamoto, che sviluppò un'idea da lui stesso presentata su Internet a fine 2008. Per convenzione se il termine Bitcoin è utilizzato con l'iniziale maiuscola si riferisce alla tecnologia e alla rete, mentre se minuscola (bitcoin) si riferisce alla valuta in sé.\cite{wiki-bitcoin}
\\Il progetto Bitcoin è nato con l'intento di risolvere i problemi di fiducia, trasparenza e responsabilità tra due parti nello scambio di denaro per beni e servizi su Internet, senza l’utilizzo di intermediari. Bitcoin, infatti, rappresenta la prima rete di pagamento che utilizza una tecnologia distribuita peer-to-peer per operare senza un'autorità centrale: la gestione delle transazioni e l'emissione di denaro sono effettuati collettivamente dalla rete.
\\La rete Bitcoin utilizzando la tecnologia blockchain, è una catena di blocchi concatenati tra di loro, infatti ogni blocco contiene il riferimento al hash del blocco precedente. Questo sistema è immutabile perchè se si volesse modificare un blocco si dovrebbero modificare tutti i blocchi successivi ad esso, per fare ciò si dovrebbe creare una catena di blocchi che è più lunga di quella esistente. Ovviamente, la distribuzione e la natura delle tempistiche del processo rendono praticamente impossibile che ciò accada.
\\Inotre, utilizzando un sistema distribuito, permette di tenere traccia di tutti i trasferimenti in modo da evitare il problema del double spending (spendere due volte). Tutti gli utenti sono a conoscenza di ciò che accade e pertanto non c’è bisogno di una autorità centrale che gestisca le transazioni.
\\In aggiunta, Bitcoin consente il possesso e il trasferimento anonimo delle monete, ed infatti i dati necessari per usufruire dei propri bitcoin sono salvati in uno o più personal computer sotto forma di digital wallet (portafoglio). I bitcoin sono trasferiti tramite Internet a chiunque disponga di un indirizzo bitcoin. La struttura peer-to-peer della rete e l’assenza di un ente centrale rende impossibile a qualsiasi autorità il blocco dei trasferimenti, il sequestro dei bitcoin senza il possesso delle relative chiavi o la svalutazione dovuta all’immissione di nuova moneta. Per questo motivo i bitcoin sono la moneta più utilizzata nel Deep Web.
\\
\\Ricapitolando, le principali caratteristiche su cui si basa questo sistema di enorme successo sono le seguenti:
\begin{itemize}
\item \textbf{Nessuna autorità centrale}: non dipende da nessuna terza parte privata o ente governativo, e, pertanto, il valore dei BTC è liberamente contrattato sul mercato. Si tratta quindi di un sistema decentralizzato;
\item \textbf{Irreversibile e non falsificabile}: una volta che una transazione è stata effettuata ed è inclusa nella blockchain, non può più essere annullata, nemmeno dal mittente;
\item \textbf{Anonimo}: chiunque può scaricare il software ed iniziare ad effettuare transazioni, senza registrarsi, senza comunicare dati personali e senza svelare la propria identità.
\item \textbf{Sistema distribuito su rete Peer-to-peer (rete paritaria o paritetica)}: qualsiasi nodo è in grado di avviare e completare una transazione in modo autonomo.
\item \textbf{Inflazione determinata a priori}: L’emissione di nuovi BTC è determinata dall’algoritmo stesso del programma, e non può essere modificata.
\end{itemize}
Per capire meglio questi punti, verrà illustrata in modo dettagliato una transazione tra due utenti Alice e Bob che vogliono scambiarsi bitcoin.

\begin{figure}[H]
	\centering
	\includegraphics[width=0.3\textwidth]{images/bitcoinLogo.png}
	\caption{Logo Bitcoin.}
	\label{fig:bitcoinLogo}
\end{figure}

\subsection{Come avviene una transazione}
\label{sec:come avviene una transazione}
Un esepio pratico di scambio di moneta aiuterà nel capire come funziona il sistema. I nostri attori saranno Alice (pagante) e Bob (ricevente).
\\
\\I passi da eseguire sono:
\begin{enumerate}\itemsep2pt

\item Bob comanda alla sua applicazione (wallet) su PC o smartphone di creare un indirizzo. Il software restituisce una sequenza alfanumerica da 26 a 35 caratteri. Questo è un esempio: 12gXGyXWkvyDAjVKZHyGGstVYyXJ6ZjgqV

\item Bob copia l'indirizzo e lo mostra al mittente tramite qualunque mezzo: via mail, messaggio, QR code, pubblicato su un sito internet, scritto su carta, dettato al telefono etc.

\item Alice inserisce nel suo software l'indirizzo di Bob e la quantità di Bitcoin da inviare, inoltre specifica l'ammontare della commissione da pagare al minatore (detto "miner") per convalidare la transazione. Ad oggi i più comuni software stabiliscono automaticamente una commisione fissa a 0,00001 bitcoin.

\item Bob vede immediatamente sul suo software la transazione avvenuta, ma prima di considerare il pagamento effettuato attende che la transazione sia inserita nella blockchain.

\item In un massimo di 10 minuti circa la transazione viene inserita in un blocco della blockchain da parte del miner, che ha convenienza a inserirla perché in questo modo ottiene la commissione pagata da Alice.

\item Bob può sentirsi sicuro che la transazione è confermata con l’aumentare di blocchi che vengono aggiunti alla blockchain conseguentemente a quello che contiene la transazione. 

\item Bob avrà nel proprio portafoglio i bitcoin inviati da Alice per spenderli in una nuova transazione.
\end{enumerate}

\begin{figure}[H]
	\centering
	\includegraphics[width=\textwidth]{images/transazioneBitcoin.png}
	\caption{Come avviene una transazione Bitcoin.}
	\label{fig:bitcoinTransaction}
\end{figure}

\subsection{Come avviene una transazione dal punto di vista tecnico}
\label{sec:come avviene una transazione dal punto di vista tecnico}
La stessa transazione affrontata con un occhio più tecnico, soffermando l'attenzione sulla parte di creazione delle chiavi e della conferma ed aggiunta del blocco.
\\
\\I passi da eseguire sono:
\begin{enumerate}\itemsep2pt
\item Tramite il software che si occupa del wallet, Bob genera in modo del tutto casuale una chiave privata, che viene salvata sul suo computer.
\item La chiave privata viene convertita in una chiave pubblica tramite un procedimento matematico. Comunemente è il software di Bob a generare automaticamente la chiavi quando Bob chiede all'applicazione di creare un indirizzo da comunicare al mittente. In realtà Bob potrebbe utilizzare una chiave privata facilmente ricordabile, come “sum qui sum”, e ricavare Public key e indirizzo da questa. Il procedimento matematico si basa su un algoritmo chiamato “Elliptic Curve Digital Signature Algorithm”\cite{bitcoin:ECDSA} che utilizza la curva ellittica.
\\È teoricamente possibile ma statisticamente impraticabile scoprire la chiave privata partendo da quella pubblica: poiché il procedimento matematico applicato è unidirezionale, il processo inverso per indovinare la chiave privata richiederebbe una quantità di tentativi e una potenza di calcolo talmente enorme da essere al di là di ogni possibilità
\item La chiave pubblica viene a sua volta crittografata e accorciata tramite un hash. Possiamo chiamare la nuova chiave pubblica Public Key Hash. La chiave pubblica originaria invece è detta Full Public Key.
\item L'hash della chiave pubblica viene convertito in una riga di massimo 35 caratteri (per comodità pratica), che costituisce l'indirizzo del portafoglio di Bob. Per esempio: 12gXGyXWkvyDAjVKZHyGGstVYyXJ6ZjgqV.
\item Bob spedisce l'indirizzo ad Alice.
\begin{figure}[H]
	\centering
	\includegraphics[width=\textwidth]{images/passiDiBob.png}
	\caption{Azioni del ricevente.}
	\label{fig:bobSteps}
\end{figure}
\item Il software di Alice decodifica immediatamente l'indirizzo in una normale Public Key Hash
\item \label{aliceCreaTransazione} Alice crea la transazione. Si può pensare la transazione come un codice che contiene diverse informazioni, ciascuna rappresentabile come una stringa composta da molti caratteri:
\begin{itemize}
	\item l’input: uno o più output di una transazione precedente fatta nei confronti 	di Alice, da cui ella attinge i bitcoin che «spedisce» nel nuovo output
	\item l'output: la quantità di bitcoin spediti. Possono esserci più output per 			ogni transazione, ciascuno identificato con un ID specifico
	\item l'istruzione per la firma (la "signature script"): ovvero le le istruzioni 		che Bob dovrà fornire per convalidare la transazione, dimostrando di essere il 			possessore del nuovo output. È proprio per la creazione dello script che il 			software di Alice ha bisogno del Public Key Hash fornito da Bob. Le informazioni 		necessarie per validare la firma sono due, entrambe già in possesso di Bob: la 			full Public Key e la Private Key, che dovranno combaciare col Public Key Hash 			specificato da Alice nello script.
\end{itemize}
Queste informazioni vengono processate insieme nella creazione di un unico hash chiamato txid (transaction identifier)
\begin{figure}[H]
	\centering
	\includegraphics[width=\textwidth]{images/passiAlice1.png}
	\caption{Prima parte. Azioni del pagante.}
	\label{fig:aliceSteps1}
\end{figure}
\item Tutti i bitcoin che Alice ha a disposizione su un particolare input vengono "spediti" nella transazione. Infatti nella transazione è coinvolta sempre l'intera quantità di bitcoin presenti nell’input anche se Bob ne ha richiesti molti meno. Se Alice dispone di un input di 100 bitcoin e ne trasferisce 20 a Bob, l’input è sempre trasferito nella sua interezza di 100 bitcoin. In questo caso avrà due output diversi, uno di 80 bitcoin (al lordo della commissione per il miner) che tornano al portafoglio di Alice (il change output), l'altro di 20 bitcoin che vanno all'indirizzo di Bob. L’unico caso di transazione che abbia un solo input e un solo output è quello in cui l’input corrisponde esattamente all’ammontare richiesto da chi riceve i bitcoin. Spesso le transazioni hanno più output, e quindi i bitcoin trasferiti vanno ad indirizzi con diverse chiavi pubbliche e private.
\item Alice trasmette via internet al software di tutti gli altri nodi tutte le informazioni relative alla transazione. I nodi sono rappresentati da tutti coloro che hanno il software Bitcoin Core sui propri pc/dispositivi o numerosi altri software che permettono di "collegarsi" al network Bitcoin.
\begin{figure}[H]
	\centering
	\includegraphics[width=\textwidth]{images/passiAlice2.png}
	\caption{Seconda parte. Azioni del pagante.}
	\label{fig:aliceSteps2}
\end{figure}

\item I minatori inseriscono le transazioni ancora non confermate nella blockchain.
\\Per inserire le transazioni all’interno della blockchain il miner deve creare un nuovo blocco, processo che richiede una quantità di calcolo molto elevata e dunque una spesa in energia elettrica e strumenti. Un miner ha interesse a inserire quante più transazioni nel blocco che vuole creare poiché guadagnerà tutte le commissioni pagate su ciascuna transazione. Se una transazione non include alcuna commissione, il miner non ha alcun interesse economico nell’inserirla nel blocco.

\begin{figure}[H]
	\centering
	\includegraphics[width=\textwidth]{images/azioneMiner.png}
	\caption{Azione dei miner.}
	\label{fig:minerSteps}
\end{figure}

Per inserire le transazioni nel blocco, i minatori partono dagli id delle transazioni (txid), ciascuno dei quali rappresenta l’hash di tutte le informazioni inerenti una singola transazione (passo \ref{aliceCreaTransazione}).
\\I txid sono accoppiati due a due, creando un hash per ogni coppia di transazioni. Ogni hash viene poi accoppiato con un altro hash, creando un hash figlio dei due hash precedenti, e così via finché non si arriva a un unico hash. In caso di numeri dispari un hash viene processato con una sua copia identica. Questo procedimento può essere rappresentato come un albero, il cosiddetto Merkle tree [\ref{fig:merkelTree}],  dove le foglie sono le transazioni txid, i rami (biforcuti) gli hash intermedi e la radice l’hash finale, prodotto di tutti gli altri hash: la Merkle root. L’hash finale è come l’ultimo di una stirpe e porta con sé il “DNA” di tutti gli hash precedenti\cite{merkle-tree}.
\\Grazie alla struttura del Merkle tree, non è necessario conoscere tutte le transazioni incluse in un blocco per verificare che una singola transazione ne faccia parte, è invece sufficiente seguire un particolare ramo che collega una foglia (una transazione) alla merkle root.
\begin{figure}[H]
	\centering
	\includegraphics[width=\textwidth]{images/merkle_tree.png}
	\caption{Merkle tree.}
	\label{fig:merkelTree}
\end{figure}

\item Bob ora vuole spendere i suoi nuovi bitcoin in una nuova transazione, il destinatario è Charlie. Bob segue la stessa procedura che ha seguito Alice, creando una transazione specificando output, signature script (per cui gli serve il public key hash di Charlie), timestamp e versione del software.
\\Bob però deve dimostrare di essere il possessore dei bitcoin che invia a Charlie, ovvero i bitcoin presenti nell’input della transazione. Tale input è anche l’output della transazione fra Alice e Bob. Quest’ultimo per dimostrare di possedere l’ouput di quella transazione deve porre la sua firma (signature). Bob inserisce quindi la sua Full Public Key, verificando che corrisponde al Public Key Hash dato in precedenza ad Alice, e la sua Private Key, che rappresenta la conferma che Bob solo è la persona che ha originato inizialmente quella Public Key. Infatti seppur sia teoricamente possibile, è per motivi statistici «infattibile» scoprire la Private Key partendo dalla Public Key. Il procedimento di firma è del tutto automatizzato dal software Bitcoin\cite{transazioneBitcoin}.

\end{enumerate}
\section{Analisi dati su sitemi distribuiti}
\label{sec:analisi dati su sistemi distribuiti}
Il numero di utenti che ogni giorno spende bitcoin è in costante crescita. La tecnologia Bitcoin quindi è costretta a lavorare con volumi di dati sempre crescente. Per questo motivo, le applicazioni che fanno analisi devono adattarsi scegliendo sistemi consoni a queste moli di dati: i Sistemi distribuiti.
\\Esistono più definizioni (più o meno equivalenti fra loro) di un sistema distribuito, fra cui:
\begin{itemize}
\item  "Un sistema distribuito è una porzione di software che assicura che un insieme di calcolatori appaiano come un unico sistema coerente agli utenti del sistema stesso" (Maarten van Steen, 2016).\cite{sitema-distribuito-1}
\item "Un sistema distribuito consiste di un'insieme di calcolatori autonomi, connessi fra loro tramite una rete e un middleware di distribuzione, che permette ai computer di coordinare le loro attività e di condividere le risorse del sistema, in modo che gli utenti percepiscano il sistema come un unico servizio integrato di calcolo" (Wolfgang Emmerich, 1997).\cite{sitema-distribuito-2}
\item "Un sistema distribuito è un sistema in cui il fallimento di un computer che non sapevi neppure esistere può rendere il tuo computer inutilizzabile" (Leslie Lamport, 1987).\cite{sitema-distribuito-3}
\end{itemize}
Sintetizzando, un sistema distribuito è un insieme di processori indipendenti (con proprie risorse hardware/software) interconnessi da una rete di comunicazione, che cooperano per condividere alcune delle risorse ovunque distribuite ed eseguire algoritmi parallelamente. Questi sistemi riescono ad apparire all'utente come un singolo sistema, permettendo di avere una certa estrazione dall'harware dei singoli nodi.
\\I sistemi distribuiti [\ref{fig:sistemaDistr}] si contrappongo ai sistemi centralizzati, nella quale tutto il lavoro è eseguito da un solo calcolatore [\ref{fig:sistemaCentr}]. Un esempio di sistema distribuito è la rete Internet stessa, che si estende a livello mondiale comprendendo risorse fisicamente molto distanti tra loro, in cui processi con funzioni diverse e connessi da reti di vario tipo si scambiano messaggi informativi basati su disparati protocolli di comunicazione.
\begin{figure}[H]
     \subfloat[Sistema distribuito\label{fig:sistemaDistr}]{%
       \includegraphics[width=0.50\textwidth]{images/sistemaDistr.png}
     }
     \hfill
     \subfloat[Sistema centralizzato\label{fig:sistemaCentr}]{%
       \includegraphics[width=0.50\textwidth]{images/sistemaCentr.png}
     }
     \caption{Differenza tra sistema distribuito e centralizzato.}
  	 \label{fig:sistemaDisCent}
\end{figure}
L'applicazione creata è eseguita su un sistema distribuito questo perché abbiamo bisogno di tempi di risposta bassi ed alta affidabilità dei dati. Per questo motivo il codice viene partizionato in piccoli problemi ed eseguiti dai nodi del nostro cluster. Il processo di partizione dei dati viene automatizzato dal framework Spark [\ref{sec:spark}] che si occupa della gestione, invio e recupero dei dati tra i vari nodi. 

\subsection{Caratteristiche di un sistema distribuito}
\label{subsec:caratteristiche di un sistema distribuito}
Un sistema distribuito si definisce tale se rispetta alcune caratteristiche come: 
\begin{itemize}
\item \textbf{Remoto}: le componenti di un sistema distribuito devono poter essere trattate allo stesso modo sia che siano in locale che in remoto.
\item \textbf{Concorrenza}: è possibile eseguire contemporaneamente due o più istruzioni su macchine differenti.
\item \textbf{Malfunzionamenti parziali}: Ogni componente del sistema può smettere di funzionare correttamente indipendentemente dalle altre componenti del sistema; questo non deve compromettere le funzionalità dell'intero sistema. I fallimenti che possono affliggere i processi possono essere di varia natura.
\item \textbf{Eterogeneità}: un sistema distribuito è eterogeneo per tecnologia sia hardware che software. Si realizza in tutti i contesti come rete di comunicazione, protocollo di rete, linguaggi di programmazione, applicazioni, etc.
\item \textbf{Autonomia}: un sistema distribuito non ha un singolo punto dal quale può essere controllato, coordinato e gestito. La collaborazione va ottenuta inviando messaggi tra le varie componenti del sistema e gestita tramite politiche di condivisione e di accesso che devono essere rigorosamente seguite. 
\item \textbf{Evoluzione}: un sistema distribuito può cambiare sostanzialmente durante la sua vita, sia perché cambia l'ambiente sia perché cambia la tecnologia utilizzata. L'obiettivo è quello di assecondare questi cambiamenti senza costi eccessivi.\cite{distSist:caratteristiche}
\end{itemize}

\subsection{Vantaggi e Svantaggi}
\label{subsec:vantaggi e svantaggi}
I sistemi distribuiti hanno dei pro e dei contro che sono da prendere in considerazione quando si vogliono adottare soluzioni di questo tipo.  
\\I vantaggi nell'utilizzo dei sistemi sono:
\begin{itemize}
\item \textbf{Connettività e collaborazione}: possibilità di condividere risorse hardware e software (compresi dati e applicazioni)
\item \textbf{Prestazioni e scalabilità}: la possibilità di aggiungere risorse fornisce la capacità di migliorare le prestazioni e sostenere un carico che aumenta (scalabilità orizzontale)
\item \textbf{Tolleranza ai guasti}: grazie alla possibilità di replicare risorse e dati
\item \textbf{Apertura}: l’uso di protocolli standard aperti favorisce l’interoperabilità di hardware e software di fornitori diversi
\item \textbf{Economicità}: i sistemi distribuiti offrono spesso (ma non sempre) un miglior rapporto prezzo/qualità che i sistemi centralizzati basati su mainframe
\end{itemize}

Gli svantaggi invece sono:
\begin{itemize}
\item \textbf{Complessità}: i sistemi distribuiti sono più complessi di quelli centralizzati e quindi risultano più difficili da capire, inoltre, lo sviluppo delle applicazioni deve essere implementato ad hoc.
\item \textbf{Sicurezza}: l’accessibilità in rete pone problematiche di sicurezza
\item \textbf{Gestibilità}: è necessario uno sforzo maggiore per la gestione del sistema operativo e delle applicazioni
\end{itemize}

%\section{Stato d'arte}
\label{sec:stato d'arte}

\chapter{Overview e progettazione di sistema}
\label{chap:overview e progettazione di sistema}
In questo capitolo viene messo in luce lo scopo dell'elaborato realizzato con particolare attenzione alle scelte progettuali adottate e alle ricerche effettuate per implementare l'argomento trattato, onde presentare un quadro generale completo. Viene, inoltre, descritta l'architettura del software, sottolineando ed approfondendo le sue principali componenti.

\section{Scopo del progetto}
\label{sec:scopo del progetto}
Lo scopo principale del lavoro di tesi è quello di creare un sistema distribuito che permetta la visualizzazione real time, la storicizzazione e l'analisi delle transazioni  provenienti dalla blockchain di bitcoin. L'obiettivo, dunque, è quello di riuscire a creare un sistema che gestisca grandi quantità di dati in un ambiente distribuito, garantendo affidabilità e consistenza dei dati anche in caso di guasti.
\\Il sistema si rivolge ad un pubblico che vuole fare analisi delle transazioni processate dalla rete bitcoin. L'utente infatti, può controllare in real time le ultime transazioni elaborate, monitorare l'intera rete blockchain per risalire a tutta la catena di transazioni, oppure controllare gli hash (indirizzi) che hanno avuto maggior punteggio di PageRank.
\\Le ultime transazioni elaborate, hanno una serie di informazioni di dettaglio come il blocco di appartenenza, l'hash che ha generato quella transazione, il timestamp ed i destinatari dei bitcoin. Mentre per quanto riguarda il monitoraggiio dell'intera blockchain, l'utente può navigare con l'utilizzo del proprio mouse, in un grafo orientato rappresentante la storia di tutte le transazioni. Inoltre, il fruitore può controllare i nodi con il maggior punteggio di PageRank e localizzarlo all'interno del grafo.  

\section{Architettura  del progetto}
\label{sec:architettura del progetto}
Le principali scelte di progetto sono state prese coerentemente con lo stato dell'arte e si è tentato di non introdurre nuova complessità al panorama esistente, ricorrendo a tecnologie, protocolli e standard già esistenti ed affermati, senza definirne di nuovi. Quindi, l'architettura dovrà essere sufficientemente generale in modo da poter garantire nuovi sviluppi ed evoluzioni future e da non comportare l'esclusione a priori di determinate soluzioni e tecnologie. 
\\Il sistema, quindi, si divide logicamente in due moduli:
\begin{itemize}
	\item \textbf{Il Sistema distribuito (back-end)}: E' la parte non visibile all'utente. Si occupa del recupero dei dati dalla blockchain di bitcoin, della storicizzazione e della pubblicazione sui topic di kafka. Interamente scritto in Java, comprende Bitcoind, Spark, Hadoop, Neo4j, Zookeeper.
	\item \textbf{Webapp (front-end)}: E' la parte visibile all'utente finale. Si occupa della rappresentazione grafica delle transazioni. Scritto in principalmente in Javascript, utilizza la potenza di NodeJS per creare l'interfaccia grafica. Integra nel proprio ecosistema Express Handlebars, WebSocket, MaterialCSS e D3js.
\end{itemize}

\begin{figure}[H]
	\centering
	\includegraphics[width=\textwidth]{images/architetturaTesi.png}
	\caption{Architettura completa.}
	\label{fig:softwareArchitetture}
\end{figure}
 
Per la costruzione di questa infrastruttura, la prima grande sfida, è stata quella di trovare un framework o un tool che permettesse di creare e programmare su di un sistema distribuito, senza complicarci la vita. Facendo ricerche sul web, la tecnologia che più si accostava meglio al mio problema è stata Apache Spark [\ref{sec:spark}]. Questo strumento riesce a garantire a pieno i vincoli che ci siamo imposti, regalandoci la possibilità di creare un sistema distribuito su vari nodi impostando opportuni file di configurazione.
\\Risolto il problema infrastrutturale si è proceduto all'analisi dei singoli sottoproblemi. L'applicazione, per testare il carico di lavoro sul sistema, fa uso dei blocchi grezzi provenienti dal software nativo del progetto Bitcoin: Bitcoind [\ref{sec:bitcoind}]. Bitcoind è un demone che invia blocchi o transazioni (a seconda di come lo si imposta) su di una coda di tipo publisher-subscriber tramite protocollo ZeroMQ [\ref{sec:ZMQ}]. I dati, quindi, sono prelevati da Bitcoind grazie all'implementazione di un connettore creato ad hoc.
\\Ottenuti i blocchi dalla coda, è nata l'esigenza di conservare i dati ottenuti così da poterli processare ed analizzare. Fortunatamente, Spark offre una nativa collaborazione con il FileSystem distribuito Hadoop [\ref{sec:hadoop HDFS}], permettendomi di tenerli salvati su una memoria di massa distribuita.
\\Oltre ad Hadoop, i dati sono stati immagazzinati in Neo4j [\ref{sec:neo4j}]. Un database NoSQL che permette il salvataggio dei dati sottoforma di grafo, cosi da poter gestire facilmente i collegamenti tra le varie transazioni.
\\L'ultimo step, è stato quello di fare analisi delle transazioni, trovando i nodi con il maggior PageRank [\ref{sec:graphx (PageRank)}]. Anche in questo caso Spark è venuto in contro grazie al modulo GraphX, contenuto nel framework, il quale contiene algoritmi (come il PageRank) già sviluppati per l'analisi sui grafi.  
\\Una volta che il sistema distribuito è completo, non resta che mostrare i risultati ottenuti. Le scelte nel campo del front-end sono migliaia ma per semplicità ed una forte attitudine ai sistemi real-time si è preferito usare NodeJS [\ref{sec:nodejs}]. NodeJS ha dei moduli che permettono l'accesso a Kafka, il tramite tra la parte di back-end e front-end. Quindi, con NodeJS è stato creato un sito web consentendo agli utenti dal proprio browser, di visualizzare lo stato delle transazioni, i valori del PageRank e le transazioni che arrivano in real time.
\section{Sistema distribuito}
\label{sec:sistema distribuito}
\subsection{Bitcoind}
\label{sec:bitcoind}
Il primo componetene del sistema distribuito ha il compito di fornire i dati da elaborare. Questa funzione è svolta dal demone Bitcoind.
\\Bitcoind, formalmente, è un software che implementa il protocollo Bitcoin per l'utilizzo delle remote procedure call (RPC). Esso è anche il secondo client Bitcoin nella storia del network \cite{wiki:bitcoind}. Per sua natura, è eseguito come processo in background quindi l'utente per interagire con esso ha bisogno di farlo tramite una interfaccia da riga di comando chiamata \textit{bitcoin-cli}. Il demone, inoltre, funge da nodo della rete Bitcoin, infatti si sincronizza con la blockchain, verifica le transazioni ed invia blocchi. Esiste una versione anche con interfaccia grafica del demone chiamata \textit{Bitcoin-QT o Bitcoin Core}, ma per lo scopo dell'elaborato si è preferito utilizzare la versione lite per limitare l'utilizzo di risorse.
\\La versione demone, inoltre, ha il vantaggio di creare una coda ZeroMQ per la comunicazione con applicazioni esterne. Il sistema distribuito, utilizza questa funzione per recuperare i blocchi in formato grezzo (sequenza di byte) ogni qualvolta sono validati dalla blockchain.

\subsubsection{ZeroMQ}
\label{sec:ZMQ}
ZeroMQ (anche conosciuto come ØMQ, 0MQ, o zmq) è una libreria di messagistica asincrona ad alte prestazioni, destinata all'uso in applicazioni distribuite o concorrenti. Fornisce code di messaggi, ma a differenza dei middleware orientati ai messaggi, il sistema ZeroMQ può essere eseguito senza un broker di messaggi dedicato \cite{wiki:ZMQ}. In particolare, fornisce socket che trasportano messaggi atomici in diversi modi:
\begin{itemize}
\item Request-reply: Connette un insieme di clienti ad un insieme di servizi. Questo è una Remote Procedure Call (RPC).
\item Publish-subscribe: Connette un insieme di publisher ad un insieme di subscriber.
\item Push-pull (pipeline): Connette i nodi in un pattern fan-out/fan-in che può avere più passaggi e cicli. Distribuisce in maniera parallela i messaggi.
\item Exclusive pair: Collega due socket in maniera esclusiva.
\end{itemize} 

\subsection{Apache Spark}
\label{sec:spark}
Apache Spark è un framework di calcolo del cluster per l'elaborazione di dati su larga scala, progettato e implementato nel 2010 da un gruppo di ricercatori dell’Università di Berkeley a San Francisco \cite{spark:hadoop}. Questo progetto nasce dall’esigenza di migliorare le prestazioni dei sistemi distribuiti “MapReduce”. Per questo si sviluppa il concetto di Resilient Distributed Dataset (RDD), che è la teoria alla base del sistema Spark. Un RDD rappresenta un set di dati che è suddiviso in partizioni (Una tabella chiave-valore suddivisa in tante sotto-tabelle o un file diviso in tanti segmenti). Un RDD ha la proprietà di essere immutabile, cioè una volta creato non può essere cambiato se non creandone un altro. La creazione di un RDD avviene a partire dai dati su disco (presi da HDFS) o da altre fonti di dati. Una volta creato, un RDD può restare in memoria oppure può essere materializzato su disco. Ogni RDD è descritto da un set completo di metadati che consentono la ricostruzione di una delle sue partizioni in caso di fault. Spark nasce come un sistema per creare e gestire job di analisi basati su trasformazioni di RDD. Dato che gli RDD nascono e vivono in memoria, l’esecuzione di lavori iterativi, o che trasformano più volte un set di dati, sono immensamente più rapide di una sequenza di MapReduce; questo perchè il disco non viene mai (o quasi mai) impiegato nell’elaborazione.
\\Come detto in precedenza, Spark non utilizza MapReduce come motore di esecuzione; invece, utilizza il proprio runtime distribuito (DAG) per l’esecuzione di jobs su un cluster. Quando viene invocata un’azione su un RDD, viene creato un “job”. Un Directed Acyclic Graph o DAG è un grafo aciclico in cui ogni nodo è una partizione di RDD e ogni vertice è una trasformazione. 
\\A differenza di MapReduce, il motore DAG di Spark può processare pipeline arbitrarie di operatori e tradurle in un unico “job” per l'utente. Spark sta dimostrando di essere una buona piattaforma su cui costruire strumenti di analisi, e, anche a questo fine del progetto Apache, Spark include moduli per: machine learning (MLlib), elaborazione grafica (GraphX), elaborazione di stream (Spark Streaming) e SQL (Spark SQL) \cite{spark:hadoop}.
\\Spark Streaming è un'estensione dell'API Spark di base che consente l'elaborazione in streaming scalabile, ad alto throughput e con tolleranza agli errori dei flussi di dati in tempo reale. I dati possono provenire da molte fonti come Apache Kafka, che analizzeremo più avanti, Apache Flume, che è un servizio distribuito affidabile per raccogliere, aggregare e spostare in maniera efficiente grandi quantità di dati di log, Twitter o socket TCP e possono essere elaborati utilizzando algoritmi complessi espressi con funzioni di alto livello come map, reduce, join e window. Infine, i dati elaborati possono essere trasferiti a filesystem, database e dashboard live. Spark Streaming riceve streams di dati di input e li divide in batch, che vengono quindi elaborati dal motore Spark e da lì verrà generato il risultato, ossia lo stream finale \cite{spark:home}.

\subsubsection{Hadoop HDFS}
\label{sec:hadoop HDFS}

\subsubsection{Neo4j}
\label{sec:neo4j}

\subsubsection{Graphx (PageRank)}
\label{sec:graphx (PageRank)}

\subsection{Zookeeper}
\label{sec:zookeeper}
ZooKeeper è un servizio, centralizzato e open source, di coordinamento affidabile per applicazioni distribuite \cite{zookeeper:home}. I servizi di coordinamento sono notoriamente difficili da ottenere; sono, infatti, particolarmente inclini a errori derivanti, ad esempio, da condizioni di stallo. ZooKeeper, invece, fornisce servizi operativi per cluster di grandi dimensioni. Esempi di questi servizi includono: informazioni di configurazione distribuita, un servizio di sincronizzazione e un registro di denominazione per i sistemi distribuiti. Le applicazioni sfruttano questi servizi per coordinare l'elaborazione distribuita tra cluster di grandi dimensioni. ZooKeeper inoltre, mira a semplificare la natura di questi diversi servizi in un'interfaccia molto semplice creando un servizio di coordinamento centralizzato. Il servizio stesso è distribuito e altamente affidabile. Questo perché, questo tipo di servizi risulta essere di difficile implementazione per le applicazioni distribuite.
\\Lo spazio dei nomi è costituito da registri di dati, chiamati znodes, in linguaggio ZooKeeper, che sono simili ai file e alle directory. A differenza di un tipico file system, progettato per l'archiviazione, i dati di ZooKeeper vengono conservati in memoria, il che significa che ZooKeeper può ottenere alti throughput e bassi numeri di latenza. Il database replicato di ZooKeeper comprende un albero di znodi, entità che rappresentano approssimativamente i nodi del file system (simili a directory). Ogni znode può essere arricchito da una matrice di byte, che memorizza i dati. Inoltre, ogni znode può avere altri znodi sotto di esso, praticamente formando un sistema di directory interno.
\\Zookeeper è considerato un servizio robusto, dal momento che i dati persistenti sono distribuiti tra più nodi (questo insieme di nodi è chiamato "ensemble") e un client si connette a uno di essi (cioè un "server" specifico), migrando se un nodo fallisce; Finché funziona la stragrande maggioranza dei nodi, l'insieme dei nodi ZooKeeper è vivo. In particolare, un nodo master viene scelto dinamicamente per consenso all'interno dell'ensemble; se il nodo principale non funziona, il ruolo del master passa a un altro nodo. È interessante notare che il nome di uno znode può essere sequenziale, il che significa che il nome che il client fornisce quando si crea lo znode è solo un prefisso: il nome completo è dato anche da un numero sequenziale scelto dall'ensemble. Ciò è utile, ad esempio, per scopi di sincronizzazione: se più client desiderano ottenere un blocco su una risorsa, possono contemporaneamente creare uno znode sequenziale su una posizione: chi ottiene il numero più basso ha diritto al blocco.
\\Inoltre, uno znode può essere “effimero” (temporaneo): questo significa che viene distrutto non appena il client che lo ha creato si disconnette. Ciò è utile soprattutto per sapere quando un client fallisce, il che può essere rilevante quando il client stesso ha delle responsabilità che dovrebbero essere prese da un nuovo client. Tutte le richieste di scrittura dei client vengono inoltrate a un singolo server, chiamato leader: in questo modo è possibile garantire che le scritture siano mantenute in ordine, vale a dire che le scritture siano lineari. Il resto dei server ZooKeeper, chiamati follower, riceve proposte di messaggi dal leader e concorda la consegna dei messaggi. Il livello di messaggistica si occupa di sostituire i leader in caso di fallimento e di sincronizzare i follower con il leader. ZooKeeper utilizza un protocollo di messaggistica atomica personalizzato. Poiché il livello di messaggistica è atomico, ZooKeeper può garantire che le repliche locali non divergano mai. Quando il leader riceve una richiesta di scrittura, calcola lo stato del sistema quando deve essere applicata la scrittura e lo trasforma in una transazione che cattura questo nuovo stato.
\\Ogni volta che un client scrive sull'ensemble, la maggior parte dei nodi mantiene l'informazione: questi nodi includono il server per il client e ovviamente il leader. Ciò significa che ogni scrittura rende il server aggiornato con il leader. Significa anche, tuttavia, che non è possibile avere scritture simultanee. La garanzia delle scritture lineari è la ragione del fatto che ZooKeeper non offra prestazioni ottimali per i carichi di lavoro dominanti dalla scrittura. In particolare, non dovrebbe essere usato per l'interscambio di grandi dati, come i media. Fintanto che la comunicazione coinvolga dati condivisi, ZooKeeper è utile. Quando i dati possono essere scritti contemporaneamente, ZooKeeper interviene, perché impone un rigoroso ordine di operazioni anche se non strettamente necessario dal punto di vista di chi scrive. Per la lettura, invece, ZooKeeper eccelle, questo perché le letture sono simultanee poiché sono servite dallo specifico server al quale il client si connette.
\begin{figure}[H]
\centering
\includegraphics[width=\textwidth]{./images/zookeeper.png}
\caption{Funzionamento di Zookeeper.}
\label{fig:zookeeper}
\end{figure}
ZooKeeper, in sintesi, è servizio molto veloce e molto semplice. La velocità di zookeeper è data da carichi di lavoro in cui le letture dei dati sono più comuni delle scritture. Il rapporto di lettura / scrittura ideale è di circa 10:1. Inoltre, Zookeeper mantiene uno spazio dei nomi gerarchico standard, simile a file e directory, questo lo rende un sistema sostanzialmente semplice. ZooKeeper viene replicato su un insieme di host (chiamato “ensemble”) e i server sono a conoscenza l'uno dell'altro. Finché sarà disponibile una massa critica di server, sarà disponibile anche il servizio ZooKeeper. Non c'è un singolo punto di errore. Per questo Zookeeper è detto: affidabile.
\\Dal momento che il suo obiettivo, però, è quello di essere una base per la costruzione di servizi più complicati, come la sincronizzazione, fornisce un insieme di garanzie tra cui:
\begin{itemize}
\item \textit{Consistenza sequenziale}: gli aggiornamenti da un client verranno applicati nell'ordine in cui sono stati inviati.
\item \textit{Atomicità}: gli aggiornamenti hanno esito positivo o negativo. Nessun risultato parziale.
\item \textit{Immagine del sistema singolo}: un client vedrà la stessa vista del servizio indipendentemente dal server a cui si connette.
\item \textit{Affidabilità}: una volta applicato un aggiornamento, esso persisterà da quel momento fino a quando un client non sovrascriverà l'aggiornamento.
\item \textit{Tempestività}: la visualizzazione dei client del sistema è garantita per essere aggiornata entro un certo limite di tempo.
\end{itemize}

\subsubsection{Kafka}
\label{sec:kafka}
Apache Kafka è una piattaforma di streaming open source distribuita che può essere utilizzata per compilare applicazioni e pipeline di dati in streaming in tempo reale. Kafka offre anche una funzionalità di broker di messaggi simile a una coda di messaggi, dove è possibile pubblicare e sottoscrivere flussi dei dati denominati \cite{kafka:microsoft}. Apache Kafka permette la gestione di centinaia di megabyte di traffico in lettura e scrittura al secondo da parte migliaia di Client. È stato sviluppato per la prima volta su LinkedIn e viene spesso utilizzato in combinazione con Apache Spark Streaming per l'elaborazione dei flussi in tempo reale.
\\Kafka ha un'architettura che differisce significativamente da altri sistemi di messaggistica. Ogni nodo prende il nome di broker. Kafka offre un'API Producer per la pubblicazione di record in un topic Kafka e un’API Consumer che viene utilizzata quando si sottoscrive un topic. Un topic è un nome di categoria o feed a cui i record sono pubblicati. I Topic in Kafka sono sempre multi-abbonato; cioè, un argomento può avere zero, uno o più consumatori che si abbonano ai dati scritti su di esso \cite{kafka:home}. Kafka archivia i record in topic. Un topic è un insieme di messaggi della stessa categoria; per ogni topic il cluster Kafka mantiene un registro partizionato. Ogni partizione è una sequenza ordinata ed immutabile di messaggi che vengono aggiunti in continuazione all’interno del registro. I record, che come abbiamo visto sono archiviati in topic, vengono prodotti da un producer e usati dai consumer. I Producers o produttori pubblicano i messaggi (i dati) all’interno di un topic. Il Producer è responsabile di scegliere in quale partizione del registro del topic inserire un proprio messaggio. Ogni Producer sceglie il proprio algoritmo di assegnamento (per esempio un semplice round robin). I Consumers o consumatori leggono (consumano) i dati presenti all’interno del topic. Nel caso del sistema distribuito il produttore è integrato nell'ecosistema Spark, mentre il consumatore è una libreria creata ad-hoc presente all'interno della Webapp, in particolare nell'ecosistema NodeJS. 
\\Ogni partizione è una sequenza ordinata e immutabile di record che viene continuamente aggiunta a un registro di commit strutturato. Ai record nelle partizioni viene assegnato un numero ID sequenziale chiamato offset che identifica in modo univoco ogni record all'interno della partizione \cite{kafka:home}.
\\In Kafka le partizioni sono replicate tra i nodi per garantire protezione in caso di interruzioni dei nodi (broker). La replicazione di una partizione viene gestita automaticamente in modo che sia assegnata a diversi brokers. Kafka elegge per ogni Broker una partizione “Leader” (indicato con (L)) e tutte le scritture e le letture dovranno passare alla partizione “Leader” scelta. Il traffico dei producer viene indirizzato al leader di ogni nodo, usando lo stato gestito da ZooKeeper. 
\begin{figure}[H]
\centering
\includegraphics[width=\textwidth]{./images/kafkaArchitetture.jpg}
\caption{Come funziona Kafka}
\label{fig:clusterKafka}
\end{figure}
Infine, la messaggistica in Kafka prevede solo due di tipi di modelli: 
\begin{itemize}
\item \textit{Queuing (Coda)}: un pool di Consumers può leggere dal Server e ciascuno può leggere i dati solamente durante il suo turno; 
\item \textit{Publish - subscribe}: il messaggio viene trasmesso a tutti i Consumers. 
\end{itemize}
Kafka offre una sola implementazione dell'entità Consumer che generalizza entrambi i modelli, il consumer group.  Un Consumer etichetta se stesso con il nome del gruppo a cui decide di far parte e ciascun messaggio pubblicato in un topic, seguito dal gruppo, viene consegnato a ciascun Consumer presente.  Se tutti i Consumer hanno lo stesso consumer group, il sistema si reduce ad una semplice coda con priorità first in - first out (FIFO).  Se tutti i consumer hanno un consumer group diverso, il sistema automaticamente diventa di tipo publish-subscribe.

\section{WebApp}
\label{sec:webapp}
\begin{figure}[H]
	\centering
	\includegraphics[width=\textwidth]{images/webApp.png}
	\caption{Architettura in dettaglio della webapp.}
	\label{fig:webAppArchitetture}
\end{figure}
\subsection{Node.js}
\label{sec:nodejs}
Node.js (anche conosciuto come Node o Nodejs) è un framework/piattaforma molto potente basato sul motore \textit{JavaScript V8} di Google Chrome, per creare facilmente applicazioni di Web veloci e scalabili. Pubblicata da Ryan Dahl nel 2009, viene utilizzato per sviluppare applicazioni Web intensive di I/O come siti di streaming video, applicazioni a pagina singola e altre applicazioni Web. Node.js è un ambiente open source, multi-piattaforma per lo sviluppo di applicazioni lato server completamente gratuito, utilizzato da migliaia di sviluppatori in tutto il mondo. Le applicazioni Node.js sono scritte in Javascript e possono essere eseguite all'interno del runtime di Node.js su OSX, Microsoft Windows e Linux. 
\\Utilizza un modello I/O non bloccante e basato sugli eventi che lo rende leggero ed efficiente, perfetto per applicazioni in tempo reale ad alta intensità di dati che funzionano su dispositivi distribuiti \cite{node:wiki}. Il modello di networking su cui si basa Node.js non è quello dei processi concorrenti, ma I/O event-driven: ciò vuol dire che Node richiede al sistema operativo di ricevere notifiche al verificarsi di determinati eventi, e rimane quindi in sleep fino alla notifica stessa. Solo in tale momento torna attivo per eseguire le istruzioni previste nella funzione di \textit{callback}, così chiamata perché da eseguire una volta ricevuta la notifica, che il risultato dell'elaborazione è disponibile. Tale modello di networking, implementato anche nella libreria Event machine per Ruby e nel framework Twisted per Python, è ritenuto più efficiente nelle situazioni critiche in cui si verifica un elevato traffico di rete \cite{node:wiki}. Di fronte alle esigenze di migliorare le performance dei software di rete Ryan Dahl ha creato una piattaforma che esegue le operazioni di I/O particolarmente lente (comunicazioni di rete o accesso al disco) in modo asincrono, rendendo la programmazione su Node JS diversa da qualsiasi esperienza con altri linguaggi. In definitiva, l'obiettivo di Node.js è quello di fornire un modo veloce per realizzare applicazioni web scalabili in termini di gestione delle connessioni da parte dei client verso il web server.
\\Nel momento in cui una operazione di I/O considerata lenta (di solito lo è se riguarda la rete o il disco fisso) viene eseguita da un programma in Node JS, V8 si occupa di trasferire la chiamata su un thread non bloccante fra quelli che ha a disposizione nella sua \textit{thread-pool base}. In questo modo, il thread principale con il codice può continuare la sua esecuzione senza context switch. Nel momento in cui una operazione collegata ai thread non bloccanti è terminata il kernel segnala che questo thread può tornare in coda di esecuzione. A questo punto però V8 si occuperà di intercettare il messaggio, mettere nella propria coda di esecuzione la funzione di callback specificata con l'operazione di I/O terminata e di rimettere il thread non bloccante a disposizione per altre operazioni di I/O. Così facendo, virtualmente il thread che esegue codice non si ferma mai, avvicendando le funzioni di callback delle varie operazioni terminate [\ref{fig:nodeEvent}].
\begin{figure}[H]
	\centering
	\includegraphics[width=\textwidth]{images/nodeEvent.png}
	\caption{Gestione eventi con Node.js.}
	\label{fig:nodeEvent}
\end{figure}
Bisogna però sempre tenere a mente, quando si progetta di scrivere un programma con Node.js, che questa architettura ha un effetto collaterale molto pesante; le operazioni che occupano per lungo tempo il thread in cui viene eseguito il codice (operazioni di calcolo onerose) bloccano l'interno software. Per questo motivo Node.js è assolutamente sconsigliato in caso di operazioni di calcolo complesse e nella fase di progettazione di programmi che usano questa piattaforma è sempre necessario utilizzare questa caratteristica come criterio fondamentale per la scrittura del codice.
Di seguito sono alcune delle caratteristiche importanti che rendono Node.js la prima scelta di architetti del software:
\begin{itemize}
\item \textbf{Asynchronous and Event Driven}: Tutte le API della libreria Node.js sono asincrone, ovvero non bloccanti. Significa essenzialmente che un server basato su Node non aspetta mai che un'API restituisca i dati. Il server passa all'API successiva dopo averlo chiamato ed un meccanismo di notifica di eventi consente al server di ottenere una risposta dalla precedente chiamata.
\item \textbf{Molto veloce}: Essendo costruito sul motore Javascript V8 di Google Chrome, Node.js è molto veloce nell'esecuzione del codice.
\item \textbf{Thread singolo ma altamente scalabile}: Node utilizza un singolo thread con loop di eventi. Il meccanismo degli eventi aiuta il server a rispondere in modo non bloccante e rende il server altamente scalabile rispetto ai server tradizionali che creano thread limitati per gestire le richieste. Quindi, Node utilizza un singolo programma con thread e lo stesso programma può fornire il servizio a un numero molto più grande di richieste rispetto ai server come Apache HTTP Server.
\item \textbf{Nessun Buffering}: Le applicazioni create con node non bufferizzano mai alcun dato. Queste applicazioni generano semplicemente i dati in blocchi.
\item \textbf{Licenza}: Node è rilasciato sotto licenza MIT. 
\end{itemize}
Esiste un mondo attorno a Javascript composto da librerie che ne estendono le funzionalità. Stesso discorso vale per Node.js, in quanto è attiva una comunità di sviluppo che ha realizzato in questi anni molte librerie per realizzare particolari tipi di supporto (database, Network, . . . ) ed un sistema di installazione di questi moduli che si occupa anche di eventuali dipendenze: \textit{npm}. Nei prossimi capitoli saranno analizzati alcuni progetti legati a Node.js usati nella creazione della webapp.

\subsubsection{Express.js ed Handlebars}
\label{sec:express handlebars}
Express.js (anche chiamato semplicemente Express) è un framework basato su Node.js che offre un insieme robusto di utilità per realizzare agilmente un'architettura \textit{MVC (Model-View-Controller)} sul lato server di applicazioni web single-page, multi-page ed ibride.
\\Basato sul modulo di Node chiamato \textit{connect}, risulta essere un ottimo "connettore" o \textit{middleware} tra le diverse librerie che possono popolare una webapp, tra cui WebSocket, Passport, Mustache.js, Handlebars, etc.
\\Le funzionalità offerte sono il \textit{routing}, la possibilità di gestire le configurazioni dell'applicazione ed un motore di templating.
\\Al centro di questa libreria c'è il concetto di flusso di funzioni, o come piace dire a certe persone, un set di \textit{livelli di funzione}. Infatti, per creare un'applicazione Express c'è bisogno di creare una sequenza di funzioni (livelli) che il framework può navigare. 
\\Quando una di queste decide di entrare in gioco, può completare il processo e fermare la sequenza. Dopodiché Express riprende a scorrere la sequenza fino alla fine.
\\Queste funzioni hanno delle callback associate che sono eseguite nel momento in cui un client effettua una richiesta. Questo processo prende il nome di Routing.
\\Il routing, dunque, è una funzionalità di Express.js, che determina la risposta ad una richiesta client ad un endpoint particolare; il quale può essere un URI (o percorso) o un metodo di richiesta HTTP specifico. In pratica, con Express, si possono gestire le richieste e le risposte \textit{in the middle}, cioè fra server e client. Quando il server riceve una richiesta HTTP la racchiude all'interno di un oggetto \textit{ServerRequest}. Questo oggetto, insieme all'oggetto \textit{ServerResponse}, viene passato al primo middleware che ne può modificare il contenuto, o aggiungere proprietà. Una volta terminata la modifica, il middleware richiamerà il successivo nell'eventuale catena (di funzioni) presente.
\begin{figure}[H]
	\centering
	\includegraphics[width=\textwidth]{images/ExpressRoute.png}
	\caption{Visualizzazione della catena di funzioni per rispondere ad una richiesta}
	\label{fig:expressFlow}
\end{figure}
Se l'ultima funzione della catena deve restituire un HTML come risultato, Express chiama in causa il \textit{template engine}. Questo motore, infatti si occupa di fare il tramite tra i dati presenti nel layer model e quello di presentazione. In particolare elabora la richiesta producendo, in fase di run-time, un HTML dinamico partendo da una struttura definita (template). Esistono diversi motori che possono essere utilizzati con Express ognuno con le proprie particolarità e specifiche. 
\\Per questo elaborato la scelta del templating engine è caduta su \textit{Handlebars.js}. La libreria di templating Handlebars consente di creare un'interfaccia utente ricca includendo HTML statico e contenuto dinamico, che possono essere specificati nelle doppie parentesi graffe. Handlebars.js è molto popolare, semplice da usare e con una grande community. È basato sul linguaggio dei modelli di Mustache, ma lo migliora in molti aspetti. Con Handlebars, si può separare la generazione di HTML dal resto del JavaScript e scrivere codice più pulito. Inoltre, aggiunge costrutti (if e cicli for) che permettono di creare dinamicamente l'HTML. Infine, introduce un sistema di \textit{partial} che permette allo sviluppatore di inserire nelle proprie pagine, porzioni di HTML provenienti da file esterni. 
\\Express, quindi, per costruire la pagina di risposta da inviare al client, chiama Handlerbars che preleva i file con estensione \textit{.handlebars} li "compila" e genera il risultato finale.

\begin{lstlisting}[language=HTML, label=lst:HandlebarsTemplate, caption={Esempio di HTML scritto con Handlebars.}]
<div class="entry">
  <h1>{{title}}</h1>
  <h2>By {{author.name}}</h2>
  <div class="body">
    {{body}}
  </div>
</div>
\end{lstlisting} 

Il listato \ref{lst:HandlebarsTemplate} è un esempio di HTML scritto con la sintassi di Hadelbars. I valori \textit{title}, \textit{author.name} e \textit{body} saranno sostituiti, in fase di run-time, con i valori provenienti dal controller che passerà un oggetto all'engine templating. vedi listato \ref{lst:handleCode}.

\begin{lstlisting}[language=Javascript, label=lst:handleCode, caption={Esempio di variabile passata dal controller al template engine.}]
var context = {
  title: "My First Blog Post!",
  author: {
    id: 47,
    name: "Yehuda Katz"
  },
  body: "My first post. Wheeeee!"
};
\end{lstlisting} 

\subsubsection{WebSocket}
\label{sec:WebSocket}
L'applicazione web dell'elaborato di tesi, come detto in precedenza, prevede una interfaccia grafica per la visualizzazione delle ultime transazioni provenienti da Bitcoin. Per poter implementare questa funzionalità, c'è bisogno di utilizzare tecniche che permettano l'invio di dati tra client e server. Dal semplice request/response di HTTP, l'evoluzione del web e delle sue tecnologie ha portato alla nascita di nuove tecnologie per migliorare sempre di più la comunicazione remota.
\\Il modello tradizionale di comunicazione, derivato dalle specifiche standard di HTTP, prevedeva una comunicazione sincrona: in seguito a una azione dell'utente (request), il server eseguiva l'operazione richiesta e restituiva il risultato (response). Dopo la richiesta iniziale, il client si poneva in uno stato di attesa fino a quando la risposta non era ricevuta, risultando in uno spreco di tempo e risorse. Il refresh della pagina peggiorava inoltre la user-experience. Allo stesso tempo, il server non manteneva nessuna informazione riguardo alla comunicazione appena avvenuta. Più richieste della stessa operazione dunque venivano ogni volta re-processate e rigenerate per ogni client che le richiedeva. Una comunicazione di questo tipo, semplice da effettuare dal punto di vista implementativo, risulta tuttavia inefficiente e inadatta ad applicazioni di larga scala moderne.
\\Il primo fondamentale passo è stato rendere la comunicazione da sincrona ad asincrona. Ciò è stato possibile attraverso l'uso di plugin esterni, come Flash, oppure tramite nuovi meccanismi come Ajax. Tuttavia, entrambi presentano dei problemi: nel primo caso, un utente poteva non aver intenzione di installare software esterno, rendendo così inutile il plugin; nel secondo caso, la gestione stessa della comunicazione attraverso Javascript poteva velocemente raggiungere livelli di complessità non accettabili per grandi applicazioni. Per questi motivi la comunicazione tra client e server distribuito, è implementata con l'utilizzo della nuova tecnologia associata ad HTML5: i \textit{WebSocket}.
Formalmente, WebSocket è una tecnologia web che fornisce canali di comunicazione full-duplex attraverso una singola connessione TCP. L'API del WebSocket è stata standardizzata dal W3C e il protocollo WebSocket è stato standardizzato dall'IETF come RFC 6455 \cite{websocket:wiki}.
\\Dunque, tra le novità portate da HTML5, i WebSocket rappresentano quella di maggior importanza dal punto di vista dell'interazione tra client e server. WebSocket è una tecnologia per effettuare comunicazioni bidirezionali in tempo reale. Essi prevedono un canale di comunicazione sempre attivo, a bassa latenza, tra client e server, utilizzabile da entrambi sia in scrittura che in lettura. Tale canale è costituito da una connessione TCP persistente, garantito da un handshaking client-key iniziale ed un modello di sicurezza originbased. Per la protezione dei dati trasmessi contro lo sniffing sono applicate apposite maschere. 
\\Le caratteristiche principali dei WebSocket:
\begin{itemize}
\item \textbf{Bidirezionali}: Quando il canale di comunicazione è attivo, sia il client che il server sono connessi ed entrambi possono inviare e ricevere messaggi.
\item \textbf{Full-duplex}: Dati inviati contemporaneamente dai due attori (client e server) non generano collisioni e vengono ricevuti correttamente.
\item \textbf{Basati su TCP}: Il protocollo usato a livello di rete per la comunicazione è il TCP, che garantisce un meccanismo affidabile (controllo degli errori, re-invio di pacchetti persi, ecc) per il trasporto di byte da una sorgente a una destinazione.
\item \textbf{Client-key handshake}: All'apertura di una connessione, il client invia al server una chiave segreta di 16 byte codificata con base64. Il server aggiunge a questa un'altra stringa, detta \textit{magic string}, specificata nel protocollo (“258EAFA5-E914-47DA-95CA-C5AB0DC85B11”), codifica con SHA1 e invia il risultato al client. Cosi facendo, il client può verificare che l’identità del server che ha risposto corrisponda a quella desiderata.
\item \textbf{Sicurezza origin-based}: Alla richiesta di una nuova connessione, il server può identificare l'origine della richiesta come non autorizzata o non attendibile e rifiutarla.
\item \textbf{Maschera dei dati}: Nella trama iniziale di ogni messaggio, il client invia una maschera di 4 byte per l'offuscamento. Effettuando uno XOR bit a bit tra i dati trasmessi e la chiave è possibile ottenere il messaggio originale. Ciò è utile per evitare lo sniffing, cioè l'intercettazione di informazioni da parte di terze parti.
\end{itemize}
\begin{figure}[H]
	\centering
	\includegraphics[width=\textwidth]{images/websocketProtocol.png}
	\caption{Comunicazione client/server tramite WebSocket}
	\label{fig:websocketProtocol}
\end{figure}
WebSocket è disegnato per essere implementato sia lato browser che lato server, ma può essere utilizzato anche da qualsiasi applicazione client-server. L'utilizzo dei WebSocket lato client, è possibile attraverso l'uso di specifiche API Javascript che consentono di ottenere informazioni sullo stato della connessione (aperta, chiusa, in apertura o in chiusura), di interagire con essa (inviare dati o chiudere la comunicazione) o di gestire particolari eventi come la ricezione di errori. Lato server, invece, esistono implementazioni dei WebSocket per la maggior parte dei linguaggi più utilizzati (Node.js, Java, C\#, Python, Ruby).

\subsubsection{MaterializeCSS}
\label{sec:materialCSS}
L'interfaccia grafica, \textit{GUI (Graphical User Interface)} o \textit{UI (User Interface)}, di una applicazione web determina, in molti casi, il suo successo. Progettare e programmare bene questo componente, quindi, ha un peso molto importante sulla buona riuscita anche del sistema sottostante che lo alimenta. 
\\Materialize CSS è una libreria di componenti UI creata con CSS, Javascript e HTML che permette di creare interfacce in \textit{Material Design}. Creato e progettato da Google, Material Design è un linguaggio di progettazione che combina i principi classici del design di successo con l'innovazione e la tecnologia. Sebbene si possa scegliere tra milioni di librerie, la scelta di questo tool risulta molto promettente ed è una valida alternativa a prodotti affermati, quali Bootstrap o Foundation, con i quali è molto più complicato produrre un'interfaccia con una vera identità. MaterializeCSS offre layout, animazioni e molti altri componenti che rendono semplice la creazione di pagine web. I componenti aiutano a costruire pagine Web e applicazioni \textit{responsive}, consistenti e funzioni, nel rispetto dei principi di progettazione Web moderni quali portabilità del browser e indipendenza del dispositivo. Tra i componenti di maggiore rilievo troviamo: \textit{Table} utile alla creazione responsive di tabelle, \textit{NavBar} che permette la creazione di una barra di navigazione per qualsiasi dispositivo e \textit{Icons} un set di icone disegnate con il CSS, facendo a meno così delle immagini.
\\Le principali caratteristiche del framework sono:
\begin{itemize}
\item \textbf{Installazione}: L'installazione del framework è estremamente semplice, infatti bisogna solo importare i CSS e i javascript nel tag head nelle pagine html del proprio sito.
\item \textbf{Grid System}: Materialize divide il layout di un pagina web in 12 colonne. Questo sistema prende il nome di \textit{grid system} e permette di utilizzare classi \textit{container} che rendono il sito responsive e quindi utilizzabile su ogni dispositivo.
\item \textbf{Componenti}: Il framework ha una moltitudine di componenti \textit{pronti all'uso}.
\item \textbf{Temi built-in}: Nasce con una serie di temi grafici che possono essere utilizzati.
\item \textbf{Gratuito}: E' gratuito e rilasciato con licenza MIT.
\end{itemize} 
In definitiva, Materialize è nato per facilitare lo sviluppo grafico utilizzando un design sobrio e minimale. Rendendo così la realizzazione di interfacce grafiche semplice ma funzionale.

\subsubsection{D3.js}
\label{sec:d3js}
Ultimo componente di maggiore interesse è la libreria \textit{D3.js} [\ref{fig:d3Logo}]. Questa libreria è usata all'interno di MaterilizeCSS per disegnare i grafi delle transazioni. D3.js (o solo D3 per Data-Driven Documents) è una libreria Javascript scritta da Mike Bostock come progetto successore di un precedente tool di visualizzazione chiamato Protovis \cite{d3:tutorialspoint}. E’ basata sugli standard web e sfrutta appieno le tecnologie dei browser per manipolare gli elementi. Come per JQuery, si utilizza la sintassi CSS per i selettori e si applicano gli stili agli elementi tramite fogli CSS. D3 a differenza delle altre librerie di grafici, non offre un insieme di grafici già pronti all'uso, bensì un potente framework che permette di realizzare praticamente qualsiasi tipo di grafico manipolando gli elementi di una pagina web di tipo HTML, SVG o Canvas in base al contenuto di un dataset.
\\La libreria JavaScript D3, incorporata in una pagina web HTML, utilizza funzioni JavaScript prefatte per selezionare elementi del DOM, creare elementi SVG, aggiungergli uno stile grafico, oppure transizioni, effetti di movimento e/o tooltip. Questi oggetti posso essere largamente personalizzati utilizzando lo standard web dei "fogli di stile a cascata", chiamati in inglese CSS. In questo modo grandi collezioni di dati possono essere facilmente convertiti in oggetti SVG usando semplici funzioni di D3 e così generare ricche rappresentazioni grafiche di numeri, testi, mappe e diagrammi. I dati utilizzati possono essere in diversi formati, il più comune è il JSON, valori separati da virgola CSV o geoJSON, ma, se necessario, si possono scrivere funzioni JavaScript apposta per leggere dati in altri formati. 
\\Il concetto centrale del design di D3 è permettere al programmatore di usare dei selettori, come per i CSS, per scegliere i nodi all'interno del DOM Document Object Model e quindi usare operatori per manipolarli, similmente alla libreria jQuery. La selezione può essere basata su tag, elementi, classi, identificatori, attributi o punti della gerarchia. Una volta che gli elementi sono selezionati possiamo applicare operazioni su di essi. Questo comprende leggere ed impostare attributi, mostrare testi, formattare. Gli elementi possono anche essere aggiunti e rimossi. Questo processo di modifica, creazione ed eliminazione di elementi HTML, può essere eseguito in base ai set di dati forniti, che è il concetto di base di D3.js.
\\D3.js è uno dei migliori framework in circolazione di visualizzazione dei dati e può essere utilizzato per generare visualizzazioni semplici e complesse insieme all'iterazione dell'utente e agli effetti di transizione. La sua potenza è dovuta alcune caratteristiche come l'estrema flessibilità, la facilità d'utilizzo e rapidità, il supporto a grandi dataset, codice riutilizzabile, un'ampia varietà di funzioni ed infine la possibilità di associare dati ad elementi del DOM. D3, infine è un progetto open source e funziona senza plugin. Richiede molto meno codice e offre i seguenti vantaggi:
\begin{itemize}
\item Ottima visualizzazione dei dati.
\item E' modulare. Infatti si può importare, nel proprio progetto, solo una parte di D3, senza dover scaricare l'intera libreria ogni volta.
\item Facilità nella creazione di componenti per grafici.
\item Manipolazione diretta del DOM e collegamento uno a uno con i dati in memoria.
\item E' gratuito.
\end{itemize}
\begin{figure}[H]
	\centering
	\includegraphics[width=\textwidth]{images/d3Logo.png}
	\caption{Logo D3.js}
	\label{fig:d3Logo}
\end{figure}
\chapter{Scelte tecniche ed implementazione}
\label{chap:implementazione}
L'elaborato di tesi sviluppato prevede due progetti distinti. Il primo è il sistema distribuito (abbreviato in SD) ed il secondo progetto si occupa della parte visualizzazione dei risultati che prende il nome di (TORVIAMO UN NOME).
\\ Questo capitolo mette in luce le tecnologie adottate durante la creazione dei due progetti mostrando l'effettiva implementazione.  
\section{Diagramma delle classi}
\label{sec:diagramma delle classi}
Durante lo sviluppo della parte back-end, sono state create diverse classi di supporto allo sviluppo del componente. In figura \ref{fig:UMLSD}, diagramma delle classi (UML), è possibile vedere tutte le classi che entrano in gioco per la realizzazione dell'applicativo.
\\L'applicativo sviluppato ha come punto di partenza (\textit{main}) un Job di Spark. Dalla documentazione ufficiale un job di Spark è un calcolo parallelo composto da più attività che vengono generate in risposta a un'azione \cite{spark:job}. In parole povere, è una serie di istruzioni che vengono eseguite in parallelo su più macchine, ottimizzando i tempi di risposta dell'applicativo. 
\\I passi fondamentali per creare un Job di Spark sono:
\begin{itemize}
\item \textbf{Caricare le librerie}: Immettere nel proprio \textit{classpath} le librerie di Spark.
\item \textbf{Inizializzare il contesto di Spark}: La prima operazione di codice da effettuare è quella di inizializzare lo \textit{SparkContext}, in cui comunica la locazione dei cluster da utilizzare ed altre impostazioni quali il numero, i driver di connessione, etc.
\item \textbf{Parallelizzare le collezioni di dati}: I dati di input devono essere parallelizzati tramite il metodo di SparkContext \textit{parallelize}.
Con questo metodo gli elementi della collezione vengono copiati per formare un set di dati distribuito che può essere gestito in parallelo. I dati parallelizzati diventano RDD.
\item \textbf{Operazioni sugli RDD}: Ai dati parallelizzati si applicano le principali operazioni di Spark come map, filter, reduce, etc.
\end{itemize}
Nell'applicativo questi passi sono svolti dalla classe principale che contiene il main: \textit{"App"}. La classe in questione però utilizza delle classi create ad-hoc per gestire al meglio alcuni aspetti. Di seguito sono riportate le principali classi con una breve descrizione delle loro funzionalità: 
\begin{figure}[H]
	\centering
	\includegraphics[width=\textwidth, height=0.45\textheight, keepaspectratio]{images/AppUML.png}
	\caption{UML delle classi (Sistema distribuito).}
	\label{fig:UMLSD}
\end{figure}
\begin{itemize}
\item \textbf{\textit{App}}: Questa classe è il fulcro di tutto l'applicativo. Infatti, è colei che si occupa della creazione del Job di Spark, del salvataggio nelle periferiche esterne e della messa a disposizione dei dati elaborati. In particolare, la classe App è la prima ad essere eseguita lanciando il Job di Spark Streaming che costantemente attende dati provenienti da blockchain. Ottenuti i dati, li invia al cluster di Spark che li elabora li salva sul filesystem Hadoop e nella base dati Neo4j. Questo primo passaggio è poi accompagnato dall'analisi dello stato delle transazioni esistenti, effettuato tramite la libreria GraphX, la quale applica l'algoritmo del Page Rank sui nodi presenti in Neo4j. Terminata questa fase, le ultime righe di codice creano un collegamento con Kafka per inviare i dati sui topic. 
\item \textbf{\textit{Constants}}: Questa classe è utilizza per il salvataggio ed il recupero delle costanti come gli IP delle macchine utilizzate (hadoop, kafka, neo4j), nomi dei nodi per il database etc.
\item  \textbf{\textit{BitcoinTransaction}}: La classe in questione viene utilizzata da App per creare una struttura dati in memoria che semplifica la gestione delle transazioni. Infatti, essa rappresenta l'astrazione di una transazione Bitcoin.
\item  \textbf{\textit{TransactionDBWrapper}}: Come dice il nome, questa classe è un wrapper (contenitore) per le transazioni inviate a Neo4j.
\item  \textbf{\textit{TransactionDBOutput}}: Analogamente alla precedente, questa classe è utilizzata per avere una idea di nodo quando vengono recuperate le transazioni dal database.
\item  \textbf{\textit{Neo4jManager}}: La classe Neo4jManager è fondamentale per il salvataggio dei dati in Neo4j. Infatti, si occupa della creazione di una connessione con il database, delle query da eseguire su di esso e dell'estrazione dei dati.
\end{itemize}
Sul fronte visualizzazione il discorso cambia. Il linguaggio Javascript infatti, difficilmente utilizza il paradigma ad oggetti evitando quindi di creare classi. In contrapposizione però utilizza procedure che si attivano in risposta ad una azione o ad una specifica richiesta. Per questo motivo creare un diagramma delle classi non avrebbe senso.
\\ In Blockchain explorer quindi sono stati creati una serie di file Javscript che vengono richiamati all'occorrenza dal framework NodeJS. I file in questione, figura \ref{fig:rootWebApp}, sono i componenti principali che compongono la webapp. Occorre però fare una divisione del Javascript che viene eseguito lato client e quello lato Server. Infatti tutti i file contenuti nella cartella \textit{public} sono script che vengono eseguiti solo dai browser dei client, poiché sono integrati nell'HTML che il server genera dinamicamente. I restanti file invece sono eseguiti dal motore di Chrome V8 tramite NodeJS solo sul server.
\begin{figure}[H]
	\centering
	\includegraphics[width=0.5\textwidth, height=0.45\textheight, keepaspectratio]{images/webAppRoot.png}
	\caption{Alberatura file di Blochchain Explorer.}
	\label{fig:rootWebApp}
\end{figure}
Ogni file ha un funzione specifica definita di seguito:
\begin{itemize}
\item  \textbf{\textit{server.js}}: Questo è un file di inizializzazione, nella quale viene inizializzato il server in ascolto su di una porta e caricato il framework Express. Esso rappresenta dunque il punto iniziale di tutte le richieste che verranno smistate.
\item  \textbf{\textit{configuration.js}}: E' un file che contiene le principali costanti dell'applicativo come la porta su cui far partire il server, la porta della websocket ed altri settaggi utili per il funzionamento dell'applicazione.
\item  \textbf{\textit{express.js}}: Questo file è il cuore della webapp. Essa contiene tutte le \textit{route} possibili. Infatti, contiene tutta la catena di funzioni per gestire una richiesta da parte dei client. In altre parole, in questo file sono fatte le associazioni tra path richiesto dal client e callback eseguita in risposta. In questo file troviamo il codice che viene eseguito ad ogni richiesta client, dalla visualizzazione del grafo alle ultime transazioni. Le callback, per restituire ai client una valida risposta, richiamano a loro volta funzioni provenienti da altri file.
\item  \textbf{\textit{kafka.js}}: Contiene l'implementazione del Kafka Subscriber, quindi si occupa della connessione a Kafka ed alla ricezione dei dati da parte del topic.
\item  \textbf{\textit{neo4jDriver.js}}: Analogamente a Neo4jManager, viene usato per la connessione ed il recupero dei dati presenti in base dati Neo4j. In particolare, utilizza il linguaggio Cypher per ottenere i risultati dal database. 
\item  \textbf{\textit{websocket.js}}: Questo script crea un server WebSocket per permettere ai vari client di ricevere le transazioni in real-time tramite protocollo WebSocket.
\item  \textbf{\textit{Cartella views}}: In questa cartella sono presenti tutti i template utilizzati dal sito. Infatti, contiene tutti i file con estensione \textit{.handlebars} utili al framework per creare pagine HTML dinamiche. All'interno di questi template sono linkati tutti gli script e i fogli di stile presenti nella cartella \textit{public}.
\item  \textbf{\textit{Cartella public}}: Questa cartella contiene tutte le risorse statiche che vengono caricate ed eseguite dai browser. I file Javascript presenti in questa cartella stabiliscono una connessione con il server tramite WebSocket, disegnano il grafo delle transazioni dinamicamente e creano le tabelle all'interno del sito.
\\Gli script di particolare importanza sono:
\begin{itemize}
\item  \textbf{\textit{MakeGraph.js}}: Lo script contenuto in questo file, utilizza la libreria D3.js per disegnare i grafi delle transazioni presenti sul sito.
\item  \textbf{\textit{transactionTable.js}}: Questo file contiene lo script che genera le tabelle. Grazie all'utilizzo della libreria DataTable riesce a disegnare in modo efficiente tabelle anche di grandi dimensioni.
\item  \textbf{\textit{websocketClient.js}}: Come dice il nome, questo file permette la connessione col server WebSocket istanziato da NodeJS.
\end{itemize}
\end{itemize}
\section{Codice}
\label{sec:codice}
Come detto nel paragrafo precedente, la classe principale "App" contiene l'inizializzazione dell'intera applicazione:
CODICE 
\section{Github}
\label{sec:github}
COS'è GITHUB E XKE HO L HO SCELTO
\chapter{Conclusioni e sviluppi futuri}
\label{chap:conclusioni e sviluppi futuri}
COSA HAI APPRESO DURANTE LO SVILUPPO DELL'APPLICATIVO, E COME SVILUPPI FUTURI DIRE CHE MIGLIORI LA PARTE FRONT-END


%\appendix
% INCLUSIONE APPENDICI - - PERSONALIZZARE - TENERE COERENTE CON LISTA IN ALTO
%\chapter{Screen Blockchain Explorer}
\label{app:a}
Alcuni screen dell'applicazione Blockchain Explorer.

\begin{figure}[H]
	\centering
	\includegraphics[width=\textwidth, height=0.60\textheight]{images/homePage.png}
	\caption{Home page Blockchain Explorer.}
	\label{fig:homeBE}
\end{figure}

%TODO manca foto elenco transazioni
\begin{figure}[H]
	\centering
	\includegraphics[width=\textwidth, height=0.80\textheight]{images/lastTransaction.png}
	\caption{Elenco ultime transazioni.}
	\label{fig:transactionsBE}
\end{figure}


\begin{figure}[H]
	\centering
	\includegraphics[width=\textwidth, height=0.80\textheight]{images/graphView.png}
	\caption{Visualizzazione intero grafo.}
	\label{fig:graphBE}
\end{figure}

\begin{figure}[H]
	\centering
	\includegraphics[width=\textwidth, height=0.80\textheight]{images/lastGraph-1.jpg}
	\caption{Dettaglio nodo centrale con Page Rank più alto.}
	\label{fig:graph1BE}
\end{figure}

\begin{figure}[H]
	\centering
	\includegraphics[width=\textwidth, height=0.80\textheight]{images/lastGraph-2.jpg}
	\caption{Dettaglio transazione all'interno del grafo.}
	\label{fig:graph2BE}
\end{figure}

\begin{figure}[H]
	\centering
	\includegraphics[width=\textwidth, height=0.80\textheight]{images/infoTransaction2.png}
	\caption{Dettaglio transazione.}
	\label{fig:detailBE}
\end{figure}

\begin{figure}[H]
	\centering
	\includegraphics[width=\textwidth, height=0.80\textheight]{images/pageRankView.png}
	\caption{Visualizzazione intero grafo con calcolo del Page Rank sottostante.}
	\label{fig:graphPageBE}
\end{figure}


% BIBLIOGRAFIA

\section*{\refname}
\thispagestyle{plain}
\printbibliography[heading=cartaceo,nottype=online]
\addcontentsline{toc}{chapter}{\refname}
\pagebreak
\thispagestyle{plain}
\printbibliography[heading=web,type=online]
%\printbibliography



\end{document}