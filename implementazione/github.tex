\section{Github}
\label{sec:github}
GitHub è un servizio di hosting per progetti software. Il nome "GitHub" deriva dal fatto che GitHub è una implementazione dello strumento di controllo versione distribuito \textit{Git} \cite{git:wiki}, inventato da Linus Torvalds anima e per anni programmatore di Linux. 
\\ GIT quindi è un version control system (VCS), e cioè un programma che permette di tener traccia di tutte le modifiche e le evoluzione effettuate nel corso della stesura di un codice o di un qualsiasi progetto su supporto digitale.
\\ Un software VCS permette di mantenere una copia del proprio codice
sorgente, sia in locale sia in remoto, senza incorrere in un eccessivo dispendio di energie e senza deconcentrarsi eccessivamente dalla stesure del proprio testo. In linea di principio un VCS permette di tenere sotto controllo qualsiasi documento, siano essi foto, documenti realizzati con programma di videoscrittura, fogli di calcolo, ecc.
\\ Git è diverso dai sistemi di controllo delle versioni precedenti come \textit{Subversion} in quanto è distribuito piuttosto che centralizzato. La differenza maggiore tra i due sistemi è la posizione del codice (repository): 
\begin{itemize}
\item \textbf{Centralizzato}: Tutto il software viene inviato dai programmatori direttamente ad un server centrale che detiene il repository.
\item \textbf{Distribuito}: Non esiste un server centrale, ma esistono repository locali (la macchina di sviluppo) e repository remoti (GitHub).
\end{itemize}
Per questo motivo, GIT è abbastanza veloce, soprattutto dal momento che la maggior parte delle operazioni avviene sul repository locale. Tuttavia, l'utilizzo di Git aggiunge un livello di complessità. L'invio di codice al repository locale e l'invio di commit a un repository remoto sono passaggi separati.
\\ GitHub dunque è il repository remoto di Git. Molti sviluppatori utilizzano questo server poiché gratuito. Inoltre fornisce strumenti utili come il \textit{bug traking}, un tool per controllare le differenze, la cronologia dei commit ed una sezione per commenti ed suggerimenti.
\\ Per l'elaborato di tesi è stato usato questo sistema per tenere traccia delle modifiche apportate al codice. Questo ha permesso in molti casi di ripristinare una versione funzionante del software dopo l'immissione di nuove modifiche. GitHub inoltre, è stato utilizzato per permettere a chiunque volesse riprendere il progetto, modificarlo o applicare delle correzioni, può farlo semplicemente \textit{clonando} il repository.
\\I due elaborati di tesi possono essere clonati dai seguenti URL:
\begin{itemize}
\item \textbf{Sistema Distribuito}: \href{https://github.com/Antonio90/bitcoin-webapp.git}{https://github.com/Antonio90/bitcoin-webapp.git}.
\item \textbf{Blockchain Explorer}: \href{https://github.com/Antonio90/bitcoin-webapp.git}{https://github.com/Antonio90/bitcoin-webapp.git}. 
\end{itemize}