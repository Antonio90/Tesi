\chapter{Conclusioni e sviluppi futuri}
\label{chap:conclusioni e sviluppi futuri}
Il fenomeno delle monete virtuali sta avendo molto successo anche tra le persone non vicine all'informatica. Bitcoin è la prima moneta elettronica ed anche la più famosa. Fautrice di una rivoluzione nel campo della finanza, bitocoin ha  portato un innovativo sistema per la decentralizzazione e validazione delle transazioni: Blockchain. La blockchain e l'impossibilità di una sua manomissione è sicuramente l'elemento più innovativo del sistema, la cui applicazione alternativa a Bitcoin sembra in grado di rivoluzionare tutti i sistemi di gestione centralizzata a cui siamo abituati. 
\\ L'elaborato di tesi svolto aveva come obbiettivo quello di sviluppare un sistema distribuito per la gestione dei Big Data provenienti da Bitcoin. Esso è una possibile soluzione che permette di elaborare grosse moli di dati come quelle di Bitcoin. In corso d'opera inoltre, è stata pensata anche una soluzione che permettesse agli utenti meno esperti di visualizzare le transazioni ed i risultati ottenuti dall'elaborazione: Blockchain Explorer.
\\ Il sistema distribuito nei test effettuati, con carichi elevati, ha ottenuto delle ottime prestazioni. In 24 ore di utilizzo non ha mai subito dei rallentamenti nel processare i dati. Altro discorso per quanto riguarda la parte web, infatti non sono stati effettuati test di carico eccessivi, ma le tecnologie utilizzate nella costruzione garantiscono ottime performance.
\\ Tuttavia entrambi i progetti sono resi pubblici sul repository di GitHub, per eventuali suggerimenti o migliorie da apportare. Il sistema distribuito però, è stato pensato ed implementato per non essere legato solo ai Bitcoin. Infatti, Spark può essere facilmente riadattato per ricevere stream di dati diversi dai bitcoin. Questo dà al sistema distribuito una elasticità tale da poter facilmente cambiare la fonte dati ma di lasciare invariato il cluster sottostante. Questo potrebbe essere non uno sviluppo futuro ma un "improvement" da effettuare.
\\ Bitcoin, come detto in precedenza, ha lo scopo di essere anonimo. Col sistema distribuito invece questa peculiarità potrebbe venir meno. Infatti, si potrebbe sviluppare un sistema per capire chi si cela dietro un indirizzo bitcoin. Questa funzionalità potrebbe essere sviluppata collegando le transazioni salvate nel database del sistema distribuito con un algoritmo di ricerca sul web dell'indirizzo preso in esame. Infine un altro sviluppo che si potrebbe prendere in considerazione è quello di sviluppare un'interfaccia grafica per smartphone e tablet così da tener sempre a portata di mano lo stato di Bitcoin.
