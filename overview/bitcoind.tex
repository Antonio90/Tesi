\subsection{Bitcoind}
\label{sec:bitcoind}
Il primo componetene del sistema distribuito ha il compito di fornire i dati da elaborare. Questa funzione è svolta dal demone Bitcoind.
\\Bitcoind, formalmente, è un software che implementa il protocollo Bitcoin per l'utilizzo delle remote procedure call (RPC). Esso è anche il secondo client Bitcoin nella storia del network \cite{wiki:bitcoind}. Per sua natura, è eseguito come processo in background quindi l'utente per interagire con esso ha bisogno di farlo tramite una interfaccia da riga di comando chiamata \textit{bitcoin-cli}. Il demone, inoltre, funge da nodo della rete Bitcoin, infatti si sincronizza con la blockchain, verifica le transazioni ed invia blocchi. Esiste una versione anche con interfaccia grafica del demone chiamata \textit{Bitcoin-QT o Bitcoin Core}, ma per lo scopo dell'elaborato si è preferito utilizzare la versione lite per limitare l'utilizzo di risorse.
\\La versione demone, inoltre, ha il vantaggio di creare una coda ZeroMQ per la comunicazione con applicazioni esterne. Il sistema distribuito, utilizza questa funzione per recuperare i blocchi in formato grezzo (sequenza di byte) ogni qualvolta sono validati dalla blockchain.

\subsubsection{ZeroMQ}
\label{sec:ZMQ}
ZeroMQ (anche conosciuto come ØMQ, 0MQ, o zmq) è una libreria di messagistica asincrona ad alte prestazioni, destinata all'uso in applicazioni distribuite o concorrenti. Fornisce code di messaggi, ma a differenza dei middleware orientati ai messaggi, il sistema ZeroMQ può essere eseguito senza un broker di messaggi dedicato \cite{wiki:ZMQ}. In particolare, fornisce socket che trasportano messaggi atomici in diversi modi:
\begin{itemize}
\item Request-reply: Connette un insieme di clienti ad un insieme di servizi. Questo è una Remote Procedure Call (RPC).
\item Publish-subscribe: Connette un insieme di publisher ad un insieme di subscriber.
\item Push-pull (pipeline): Connette i nodi in un pattern fan-out/fan-in che può avere più passaggi e cicli. Distribuisce in maniera parallela i messaggi.
\item Exclusive pair: Collega due socket in maniera esclusiva.
\end{itemize} 
