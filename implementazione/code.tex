\section{Codice}
\label{sec:codice}
L'elaborato di tesi, come detto in precedenza, è suddiviso in due distinti progetti: Sistema Distribuito e Blockchain Explorer. In questo capitolo, l'attenzione sarà focalizzata sulla reale implementazione, visualizzando e commentando righe di codice con maggiore interesse.
\\ Ogni applicazione Java ha un \textit{entry point} e cioè una funzione principale che viene richiamata all'avvio dell'applicativo. All'interno del sistema distribuito l'entry point è definita all'interno della classe \textit{App}. Questa classe ha il compito di:
\begin{itemize}
\item \textbf{Recupero Costanti}: Il primo passo che effettua l'applicazione è il recupero delle costanti definite all'interno di un file di \textit{properties}: \textit{bitcoin.properties}. Per leggere le costanti di questo file utilizza un metodo statico della classe \textit{PropertiesReader} chiamato \textit{readProperties}. Questo metodo [\ref{lst:prop}], prende in input un file di proprieties e restituisce una mappa chiave-valore con tutte le proprietà definite nel file.

\begin{lstlisting}[language=Java, label=lst:prop, caption={Metodo readProperties.}]
public static Map<String, String> readProperties(String propFileName){

	Map<String, String>  propMap = new HashMap<String,String>();
	Properties prop = new Properties();

	try {

		prop.load(
			PropertiesReader.class
					.getClassLoader()
					.getResourceAsStream(propFileName));

		for (String key : prop.stringPropertyNames()) {
			String value = prop.getProperty(key);
			propMap.put(key, value);
		}

	} catch (IOException e) {

		e.printStackTrace();

	}

	return propMap;
}
\end{lstlisting}

\item \textbf{Inizializzare Spark e connessione con Bitcoind}: Caricate le costanti in una mappa, non resta che inizializzare Spark. Per fare ciò, viene creato un oggetto di tipo \textit{JavaStreamingContext} [\ref{lst:intiSpark}].

\begin{lstlisting}[language=Java, label=lst:intiSpark, caption={Inizializzazione Spark Streaming.}]
JavaStreamingContext streamingContext = new JavaStreamingContext(sparkConf, new Duration(2000));
\end{lstlisting}

Questa classe si occupa di creare il contesto streaming di Spark secondo le configurazioni presenti nell'oggetto \textit{sparkConf} e di creare Job eseguiti ogni due secondi. All'oggetto \textit{sparkConf} viene dato un nome che lo identifica in caso di più Job, impostato il tipo di cluster da creare ed infine l'URL di connessione col database Neo4j [\ref{lst:sparkConf}].

\begin{lstlisting}[language=Java, label=lst:sparkConf, caption={Creazione oggetto sparkConf.}]
SparkConf sparkConf = new SparkConf().setAppName(propMap.get("appSparkName"));
					
if (!sparkConf.contains("spark.master")) {
	/*local[K] (Run Spark locally with K threads, 
	usually k is set up to match the number of 
	cores on your machine)*/
	sparkConf.setMaster("local[2]"); 
}

/**
 * Configuration of Neo4j
 * */
sparkConf.set("spark.neo4j.bolt.url", neo4jConnectionUrl);
\end{lstlisting}

Una volta che il contesto di Spark è stato creato, non resta che creare il collegamento tra il Sistema Distribuito e Bitcoind. Questa operazione viene fatta tramite l'utilizzo della libreria \textit{spark-streaming-zeromq} di Apache Bahir. La libreria in questione permette di creare una connessione tra una coda ZMQ (utilizzata da Bitcoind) e Spark [\ref{lst:bahir}].

\begin{lstlisting}[language=Java, label=lst:bahir, caption={Metodo della libreria Spark Streaming ZeroMQ.}]
JavaDStream<byte[]> lines = ZeroMQUtils.createStream(streamingContext, host, subscribe, bytesToObjects );
\end{lstlisting}

Il metodo \textit{createStream} associa, quindi, la ricezione dei dati ad una funzione di callback da richiamare per gestire i dati. In questo caso, la funzione in questione è \textit{bytesToObjects} \ref{lst:bytesToObjects}, che estrae i byte provenienti dalla coda di Bitcoind e li trasforma in oggetti parallelizzati: \textit{JavaDStream}.

\begin{lstlisting}[language=Java, label=lst:bytesToObjects, caption={Funzione bytesToObjects.}]
Function<byte[][], Iterable<byte[]>> bytesToObjects = new Function<byte[][], Iterable<byte[]>>() {
            @Override
            public Iterable<byte[]> call(byte[][] bytes) throws Exception {
                Iterable iterable = Arrays.asList(bytes[0]);
                return iterable;
            }
        };
\end{lstlisting}

\item \textbf{Salvare i dati nell'HDFS}: Ottenuti i byte provenienti da Bitcoind può partire l'elaborazione. Il primo step da effettuare è il salvataggio sul filesystem distribuito Hadoop. Questo framework, precedentemente citato, si occupa della storicizzazione distribuita dei dati. Spark, abituato a lavorare su architetture distribuite, ha al suo interno metodi nativi che permettono di salvare dati in Hadoop. Infatti è bastato chiamare il metodo \textit{saveAsTextFile} della classe \textit{RDD} per salvare i dati all'interno del filesystem.

\begin{lstlisting}[language=Java, label=lst:hadoop, caption={Salvataggio Bytes su HDFS.}]
lines.foreachRDD((bytes, time)-> {

	List<byte[]> blockAsByte = bytes.collect();
	if (!blockAsByte.isEmpty()) {
		bytes.coalesce(1).saveAsTextFile(hadoopHdfs + File.separator + "blocks" + File.separator);
	}

});
\end{lstlisting}

Il listato [\ref{lst:hadoop}] mostra come per ogni RDD viene storicizzato il contenuto sul filesystem di Hadoop.

\item \textbf{Lanciare il job di Spark}: Terminate le operazioni preliminari spark è pronto a lanciare il Job 
\begin{itemize}
\item \textbf{Trasformare i Byte in oggetti}
\item \textbf{Salvare in Neo4j}
\item \textbf{Pubblicare i dati}
\item \textbf{Calcolare PageRank}
\end{itemize}
\end{itemize} 
 
\section{Funzionamento Blockchain Explorer}
\label{sec:Funzionamento Blockchain Explorer}